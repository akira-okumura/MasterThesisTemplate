\chapter{日本語の作文入門}

\section{読者の負担を減らす}

\subsection{順接の「が」}

接続助詞の「が」には順接と逆接の2つの使用方法があります。つまり、全く異なる意味を持つわけです。長い文中にこの「が」が現れてしまうと読者には順接と逆接のどちらだと理解すれば良いのか判断するのがこんなんです。

例えば次の2つの例では、「が」が目に入った時点で即座に順接か逆接か理解できるでしょうか。

\begin{description}
\item[修正前:]「暗黒物質の存在はFritz Zwickyによって初めて提唱され、現在に至るまで複数の観測による証拠が集められてきたが、今後の観測技術の発展によってさらなる高精度観測が見込まれる。」
\item[修正後:]「暗黒物質の存在はFritz Zwickyによって初めて提唱され、現在に至るまで複数の観測による証拠が集められてきた。今後の観測技術の発展によってさらなる高精度観測が見込まれる。」
\end{description}

\begin{description}
\item[修正前:]「暗黒物質の存在はFritz Zwickyによって初めて提唱され、現在に至るまで複数の観測による証拠が集められてきたが、その直接検出には未だ成功していないため、今後の検出感度の向上が必要である。」
\item[修正後:]「暗黒物質の存在はFritz Zwickyによって初めて提唱され、現在に至るまで複数の観測による証拠が集められてきた。しかし、その直接検出には未だ成功していないため、今後の検出感度の向上が必要である。」
\end{description}

1つ目の例が順接の「が」、2つ目の例が逆接の「が」です。無闇に長い文章を避けること、順接の「が」を論文で使用するのは絶対に避けること、逆接であっても可能な限り他の接続詞に変更することを推奨します。

\subsection{「が」と「の」}

\begin{description}
\item[修正前:]「電荷\textbf{が}存在しない領域があるため」
\item[修正後:]「電荷\textbf{の}存在しない領域があるため」
\end{description}

修正前の例が日本語として間違っているわけではありません\footnote{日本語研究の分野では「ガノ交替」と呼ぶそうです。}。しかし、修正前では「領域が」まで読まないと主語が「電荷」なのか「領域」なのか判別できないため、読者に負担をかけます。

次の例のように文が長くなると、主語の判別がより一層困難になります。

\begin{description}
\item[修正前:]「レベル1トリガーとレベル2トリガー\underline{が}生成された時間差がデータ取得頻度に影響する」
\item[修正後:]「レベル1トリガーとレベル2トリガー\underline{の}生成された時間差がデータ取得頻度に影響する」
\end{description}

\subsection{前述したように、後述するように}

どの節なのかを言及せよ。また「XX 節にあるように」ではなく「XX 節で前述(後述)したように」のように書くことで、前述なのか後述なのかを読者が即座に理解できるようにせよ。読者は自分が読んでいる節番号を常に意識していないため、前述なのか後述なのかが節番号からだけでは分からない。

\subsection{不必要に否定形を使わない}

\begin{description}
\item[修正前:]「銀河中心から近くない天体の回転速度が」
\item[修正後:]「銀河中心から遠い天体の回転速度が」
\end{description}

「近い」と「遠い」のように対義語の存在する言葉の場合、不必要に否定形を使うと読者の負担が増えます。

\section{細かい日本語}

\subsection{「な」と「の」}

\begin{description}
\item[修正前:]「低質量\textbf{な}粒子」
\item[修正後:]「低質量\textbf{の}粒子」
\end{description}

\begin{description}
\item[修正前:]「高輝度\textbf{な}光源」
\item[修正後:]「高輝度\textbf{の}光源」
\end{description}

上のような例で「な」を使う人は広く見られますが、より修士論文として相応しいのは修正後の例だと奥村個人は考えています。修正前の表現に違和感を持つのは、おそらく古い世代により多いと思われます。

表現として世間一般で頻繁に使われるようになると、すなわち形容動詞として使えるほどに一般化すると、「の」から「な」への交代が進むと考えられます。物理実験の場合、普段の議論で頻出する単語の場合、その実験グループ内でのみ、この交代が起きるようです。

「低い質量な」「高い輝度な」と分解して考えると違和感が出るため、「低質量な」「高輝度な」にも違和感が出るのではないかと推測します。もし「低質量な」に違和感がないようであれば、「低体重な新生児」「低分子量な化合物」「短時間な停電」という例を考えてみると良いでしょう\footnote{Google 検索の結果では「低体重の新生児」3,360件、「低体重な新生児」1件、「低分子量の化合物」19,200件、「低分子量な化合物」3件、「低質量の粒子」10件、「低質量な粒子」1件です(2024年2月現在)。}。

この「な」と「の」の交替は真面目に考えると実は難しい問題で、両者の使い分けに厳格な規則性があるわけではなく、また一般的に使われているかどうかも判断基準になるようです\footnote{例えば「高級なワイン」では「な」を使うのが一般的だと思いますが、「上等のワイン」「上等なワイン」はどちらもよく使われるように思います。このように似たような表現であっても表記揺れが存在します。}。

例えば次の例では「の」を奥村は推奨しますが、
\begin{description}
\item[修正前:]「高圧\textbf{な}容器」
\item[修正後:]「高圧\textbf{の}容器」
\end{description}
物理を離れた次の例では、むしろ「高圧の」では違和感があります。
\begin{description}
\item[一般的表現:]「高圧\textbf{な}態度」
\item[違和感のある表現:]「高圧\textbf{の}態度」
\end{description}

また次の例では「の」でも「な」でも奥村は許容できるように思いますが、「高速な」だと違和感のある人もいるようです。
\begin{description}
\item[許容:]「高速\textbf{な}アルゴリズム」
\item[許容:]「高速\textbf{の}アルゴリズム」
\end{description}

\begin{description}
\item[許容:]「高速\textbf{な}荷電粒子」
\item[許容:]「高速\textbf{の}荷電粒子」
\end{description}

一方で、「高速」ではなく「高速度」になるとやはり「の」が良いと思います\footnote{「高い速の粒子」と分解できないからではないかと推測します。}。

\begin{description}
\item[修正前:]「高速度\textbf{な}荷電粒子」
\item[修正後:]「高速度\textbf{の}荷電粒子」
\end{description}

\subsection{「重い」と「重たい」}

「重い」は質量の大小に対する客観的な表現ですが、「重たい」はそこに人間の心理や(特に負の)感情が入り込みます。

\begin{description}
\item[修正前:]「暗黒物質候補の\textbf{重たい}粒子」
\item[修正後:]「暗黒物質候補の\textbf{重い}粒子」
\end{description}

\subsection{漢数字とアラビア数字の使い分け}

日本語文章で数字を書くときに「一つ」「ひとつ」「1つ」のどれで書くべきか悩むことがあるかもしれません。横書きの論文でこれが絶対という規則はありませんが、新聞などでも採用されている次の原則を守ると良いでしょう。

\begin{itemize}
\item 測定値や計算結果などの数値はアラビア数字で書く(例:511\,keV、273\,K)
\item 熟語として一般的になっているものは漢数字で書く(例:三角形、一般解)
\item 他の数字と入れ替えられる数値はアラビア数字で書く(例:1次近似)
\item 他の数字と入れ替えられない数値は漢数字で書く(例:四重極モーメント、数百ボルト)\footnote{「93重極モーメント」などはありませんし、「数123ボルト」のようには書けないという意味です。また、数万ボルトや数億ボルトを数10000ボルト、数100000000ボルトのように書くと違和感があると思います。}
\item 桁の大きい数値は漢数字を混ぜる(例:人口1億2000万人)
\end{itemize}

\subsection{エネルギー単位の形容詞化・名詞化}

大学院に入るとそれまでの口語表現では使わなかった雑な日本語に研究室で触れるためか、口語表現をそのまま文語に持ち込む人が多くいます。例えばエネルギーの単位であるPeVを次のように使う人がいますが、「kmの山」とか「トンの水」のような表現をしないのと同様、修正前の例は不適切です。

\begin{description}
\item[修正前:]「PeVまで宇宙線を加速する」
\item[修正後:]「PeV帯域のエネルギーまで宇宙線を加速する」
\item[修正後:]「1\,PeVまで宇宙線を加速する」
\end{description}

\begin{description}
\item[修正前:]「PeVの宇宙線」
\item[修正後:]「1\,PeVのエネルギーを持つ宇宙線」
\item[修正後:]「PeV帯域の宇宙線」
\end{description}

\section{頻繁に見かける言葉の誤用}

\subsection{予想}

\begin{description}
\item[修正前:]「ニュートリノと原子核の衝突した位置をモンテカルロデータから予想する」
\item[修正後:]「ニュートリノと原子核の衝突した位置をモンテカルロデータから\underline{推定}する」
\end{description}
\begin{description}
\item[修正前:]「超新星残骸で加速される宇宙線のエネルギー総量を理論計算により予想する」
\item[修正後:]「超新星残骸で加速される宇宙線のエネルギー総量を理論計算により\underline{求める}」
\end{description}

「予想」という言葉を、測定から確実に分かっていないものを何らかの手段で求めること全般に使用する例を見かけます。予想の「予」は「予(あらかじ)め」という意味ですから、まだ起こっていない事象を、事前に想定するような場合に用います。

また次の例では「まだ直接には発見されていないものの存在を他の観測から示す」という意味で「予想」という言葉が使われることがありますが、暗黒物質がもし実在する場合は人間がどう思おうと存在するわけですから、「あらかじめ」という意味の含まれる「予想」を使うのは不適切です。

\begin{description}
\item[修正前:]「ビリアル定理の適用により、暗黒物質の存在を予想した」
\item[修正後:]「ビリアル定理の適用により、暗黒物質の存在を\underline{示した}」
\item[修正後:]「ビリアル定理の適用により、暗黒物質の存在を\underline{指摘した}」
\item[修正後:]「ビリアル定理の適用により、暗黒物質の存在を\underline{提唱した}」
\item[修正後:]「ビリアル定理の適用により、暗黒物質の存在が\underline{要求された}」
\item[修正後:]「ビリアル定理の適用により、暗黒物質の存在\underline{量}を\underline{見積もった}」
\end{description}

\begin{description}
\item[修正前:]「超新星残骸で加速される宇宙線のエネルギー総量を理論計算により予想する」
\item[修正後:]「超新星残骸で加速される宇宙線のエネルギー総量を理論的に計算する」
\end{description}

\subsection{実施}

\begin{description}
\item[修正前:]「この実験は2020年から実施されている」
\item[修正後:]「この実験は2020年から行われている」
\end{description}

「実施」は計画を実際に行うという意味であり、現在進行形や継続の表現を伴うのは不適切です。

\subsection{形成}

「生じる」「引き起こす」などの意味で「形成する」という言葉を使う人がいますが、「形成」は形のあるものを作る場合に使用します。

\begin{description}
\item[修正前:]「トリガー信号を形成する」
\item[修正後:]「トリガー信号を\underline{生成}する」
\end{description}

\subsection{捉える}

「光を捉える」のように「検出」とすれば良いとことろをなんでもかんでも「捉える」と書く人がいますが、「捉える」には心理的な描写を含む複数の意味がありますので、曖昧な表現を避けるという観点から不適切です。

\begin{description}
\item[修正前:]「光電子増倍管でチェレンコフ光を捉える」
\item[修正後:]「光電子増倍管でチェレンコフ光を検出する」
\end{description}

\subsection{放射と放出}

「放射」や「放出」は他動詞です。

\begin{description}
\item[修正前:]「天体周辺からガンマ線が放射する」
\item[修正後:]「天体周辺からガンマ線\underline{を}放射する」
\item[修正後:]「天体周辺からガンマ線が放射\underline{される}」
\end{description}

\begin{description}
\item[修正前:]「ベータ崩壊に伴いニュートリノが放出する」
\item[修正後:]「ベータ崩壊に伴いニュートリノ\underline{を}放出する」
\item[修正後:]「ベータ崩壊に伴いニュートリノが放出\underline{される}」
\end{description}

また、「放出」は内部に元々持っていたものを外部に出すという意味で使用します\footnote{ベータ崩壊のニュートリノは元々中性子の中や原子核の中に存在していたわけではありませんが、原子核外に出すということで「放出」が一般的に使われます。}。そのため、次のような例は書き換えを推奨します。

\begin{description}
\item[修正前:]「電子が水中でチェレンコフ光を\underline{放出}する」
\item[修正後:]「電子が水中でチェレンコフ光を\underline{放射}する」
\end{description}

\subsection{統計量}

「統計量」(statistic)とは測定データ(標本)から得られる平均値や分散など、その標本に統計処理を施して得られる特徴量のことです。宇宙・素粒子系の分野では標本が大きいことを「高統計」(high statistics)と言うことがあり\footnote{これはこれで分野依存の強い方言であり、他分野には意味が通じない場合があります。}、それに引っ張られて「統計量を貯める」のように言う人がいますが、「統計量」という語は日本語の統計学用語として確立しているものですので、勝手に違う意味で使わないようにしましょう。

\begin{description}
\item[修正前:]「長時間観測によって統計量を増やし」
\item[修正後:]「長時間観測によって事象数を増やし」
\end{description}

\subsection{Root Mean Square}

二乗平均平方根(root mean square、RMS)とは、測定値$x_i$に対して次の式で定義される統計量です。

\begin{equation}
\mathrm{RMS} = \sqrt{{\frac{1}{n} \sum \limits _{i=1}^n x_{i}^2}}
\end{equation}

RMSと似たものとして、より広く使われる統計量に標準偏差$s$がありますが、RMSとは定義が異なります。

\begin{equation}
 s = \sqrt{\frac{1}{n} \sum \limits _{i=1}^n (x_i - {\overline{x}})^2}
\end{equation}

素粒子分野出身の人はRMSと標準偏差を同じ意味で使うことがありますが、これは素粒子分野で広く使われてきたデータ解析ソフトウェア内で用語の混乱があったためです。両者を混同するのは単純に間違いですので、気をつけてください。

\subsection{速さと速度}

「速さ」(speed)はスカラーであり、「速度」(velocity)はベクターです。ただし慣用的に「回転速度」(rotation speed)や「拡散速度」(diffusion speed もしくは diffusion velocity)といった用語では、スカラー量を表すにもかかわらず「速度」や velocity の使われることがあります。また「光速」(定数としての$c$)は英語では speed of light であり velocity of light とは言いませんが、日本語では慣用的に「光速度」が使われることも多々あります(「光速度不変の原理」など\footnote{特殊相対論を考えるとき、速さ(speed)は普遍ですが、速度(velocity)は普遍ではありませんので注意してください。})。

\subsection{ガンマ線観測}

「ガンマ線観測」という言葉を使用するときに、まるで観測対象がガンマ線であると勘違いし「ガンマ線を観測する」という表現をする人がいますが、観測対象はガンマ線を放射する天体です。したがって「ガンマ線で観測する」という表現が適切です\footnote{「可視光観測」という場合に、可視光自体を観測しているわけではないと考えればより分かりやすいと思います。}。

一方、これが宇宙線観測の場合になると放射天体の方向が基本的には分かりませんから、宇宙線もしくは宇宙線の引き起こす空気シャワー現象自体を観測していることになり、「宇宙線を観測する」で問題ありません。

\subsection{探索と探査}

日本語の一般的な表現では探索と探査に大きな違いはないように思いますが、物理学用語としての探索は search に対応し、探査は survey に対応すると考えれば良いでしょう。

\begin{description}
\item[修正前:]「暗黒物質の探査」
\item[修正後:]「暗黒物質の探索」
\end{description}

\begin{description}
\item[修正前:]「ガンマ線による銀河面の探索」
\item[修正後:]「ガンマ線による銀河面の探査」
\end{description}

\subsection{種族と種別}

これは誤用ではありませんが、「種族」(population)、「種別」(class)、「型」(type)という日本語は天文学用語で使い分けられている場合があるため、混乱を招きにくい単語選択をするのが良いでしょう。

\begin{description}
\item[修正前:]「銀河宇宙線の加速天体がどのような種族の天体かは未解明である」
\item[修正後:]「銀河宇宙線の加速天体がどのような種類の天体かは未解明である」
\end{description}


%\subsection{性能と特性}

%\subsection{SN比と$S/N$}

\subsection{導出}

\begin{description}
\item[修正前:]「電流と抵抗の積から電圧降下を導出した」
\item[修正後:]「電流と抵抗の積から電圧降下を計算した」
\end{description}
\begin{description}
\item[修正前:]「測定電荷量から入射光量を導出した」
\item[修正後:]「測定電荷量から入射光量を推定した」
\end{description}

おそらく「導出」という言葉になにか格好の良い響きがあるのだと思います。実際、「導出」というのは何かを論理的に導き出すという意味ですから、実際にやっていたら格好良いでしょう。この語感のために文語的な表現だと思うのか、なんでもかんでも「導出」と書く人がいます。

\subsection{データ}

\begin{description}
\item[修正前:]「図 1. に実測データを示す」
\item[修正後:]「図 1. に実測した電流と電圧の関係を示す」
\end{description}
\begin{description}
\item[修正前:]「検出器のデータを取得した」
\item[修正後:]「検出器の出力電荷の時間変化を測定した」
\end{description}

何でもかんでも「データ」と書いてはいけません。より具体的に書きましょう。

\subsection{として、としては}

\begin{description}
\item[修正前:]「光検出器として光電子増倍管を使用している」
\item[修正前:]「光検出器としては光電子増倍管を使用している」
\item[修正後:]「光検出器に光電子増倍管を使用している」
\end{description}

\begin{description}
\item[修正前:]「波形整形としては単純なRC回路による微分のみを使用した」
\item[修正後:]「波形整形には単純なRC回路による微分のみを使用した」
\end{description}

「…として(は)」は「…の立場で(は)」のような意味の助詞であり、「選手代表として宣誓する」「私としては容認できない」という使い方をします。

光電子増倍管は光検出器なのですから「光検出器として」使うのは当然であり、意味がおかしくなります。光電子増倍管が他の用途で使われうるという可能性が読者と共有される文脈であれば「として」も許容されるますが、それは物理実験の常識ではあり得ません。

\subsection{今回は}

\begin{description}
\item[修正前:]「今回はフーリエ変換を使用することで波形整形を行った」
\item[修正後:]「本研究ではフーリエ変換を使用することで波形整形を行った」
\end{description}

「今回」という言葉には現在やっている事象を指す用法がもちろんあります。しかし前回や次回などの事象が存在するかどうか読者の分からない文脈で「今回」を使用するのは不適切です\footnote{色々と試行錯誤した結果、特定の手法を採用するに至ったような場合に「今回は」とつい書きたくなるのだと思いますが、それは読者と共有できていない情報かもしれないということに気をつけましょう。}。

\subsection{における}

\begin{description}
\item[修正前:]「チェレンコフ望遠鏡アレイにおける大口径望遠鏡の開発」
\item[修正後:]「チェレンコフ望遠鏡アレイの大口径望遠鏡の開発」
\end{description}

「における」は、特定の場所などに限定するために使用する連語です。大口径望遠鏡という望遠鏡はチェレンコフ望遠鏡アレイで使用する専用の望遠鏡ですから、他の実験などから限定する必要はありません。例えば「東京における東京タワーの建設」という言い回しはできませんが、「東京における高層ビルの建設」は適切な表現です(東京タワーは東京にしかないが、高層ビルはどこにでもある)。

\subsection{イメージ}

「イメージ」という言葉は和製英語としての用法がほとんどであり、「商品写真はイメージです」のように使われます。この用例の「イメージ」は「良さそうに描かれているけれど実物は異なります」のような意味と思われます。英語本来の意味に近い用例で使われる「イメージ」は、日本語では「心象」であったり「心の中に思い描くもの」ですが、そのような意味の「イメージ」を物理系の修士論文で使用することはないでしょう\footnote{実験装置の完成想像図などで「XX 検出器の CG による完成イメージ」という用例であれば適切ですが、「完成図」と日本語で書けば十分でしょう。}。

\begin{description}
\item[修正前:]「宇宙線空気シャワーの発達イメージ」
\item[修正後:]「宇宙線空気シャワーの発達の模式図」
\end{description}

\begin{description}
\item[修正前:]「データ取得系のイメージ」
\item[修正後:]「データ取得系の概念図」
\end{description}

\subsection{波形を出力}

装置の出力信号を時間の関数としてその形状を考えたものが波形です。しかし装置は波形を出力するわけではなく、出力するのは信号です。

\begin{description}
\item[修正前:]「光電子増倍管が波形を出力し、後段の回路で記録する」
\item[修正後:]「光電子増倍管が信号を出力し、その波形を後段の回路で記録する」
\end{description}

\subsection{増倍と増幅}

「増倍」(multiply)も「増幅」(amplify)も似たような意味ですが、物理実験系では使いわけをしています。「増倍」は個数を増やす場合に、「増幅」は数えられないものを増やす場合に使用します。

\begin{description}
\item[修正前:]「光電子をダイノードで増幅する」
\item[修正後:]「光電子をダイノードで\underline{増倍}する」
\end{description}

\begin{description}
\item[修正前:]「出力電圧を後段の回路で増倍する」
\item[修正後:]「出力電圧を後段の回路で\underline{増幅}する」
\end{description}

\subsection{「なまる」と「なます」}

語感が似ているため混同する人がいますが、「なまる」(もしくは「なまらせる」)と「なます」は違う意味です。「なます」は刃物などを焼き入れした後に冷ますという意味です。

\begin{description}
\item[修正前:]「信号伝送に長いケーブルを使用したため出力波形がなまされた」
\item[修正後:]「信号伝送に長いケーブルを使用したため出力波形が\underline{なまった}」
\end{description}

\subsection{原子核を反跳する}

暗黒物質の直接探索実験における素粒子と原子核の散乱の説明の際に「原子核を反跳する」という表現を見かけますが、「反跳する」は自動詞であり他動詞ではありません。英語でも「a xenon nucleus recoils」のように書きます。

\begin{description}
\item[修正前:]「暗黒物質がキセノン原子核を反跳する」
\item[修正後:]「暗黒物質がキセノン原子核を反跳\underline{させる}」
\item[修正後:]「暗黒物質の衝突でキセノン原子核\underline{が}反跳する」
\end{description}

\subsection{光電子を検出}

光検出器で検出するのは光子であって、光電子ではありません。

\begin{description}
\item[修正前:]「SiPM で 1 光電子が検出された場合の出力波形」
\item[修正後:]「SiPM で 1 光子が検出された場合の出力波形」
\item[修正後:]「SiPM で 1 光電子が発生した場合の出力波形」
\end{description}

\begin{description}
\item[修正前:]「PMT の検出光電子数」
\item[修正後:]「PMT の検出光子数」
\end{description}

\subsection{塗布}

「塗布」という言葉は液体状のものを塗るときに使用します。烝着するときには使用しません。

\begin{description}
\item[修正前:]「真空管の内面に光電面を塗布する」
\item[修正後:]「真空管の内面に光電面を烝着する」
\end{description}

\begin{description}
\item[修正前:]「多層膜コーティングを塗布する」
\item[修正後:]「多層膜コーティングを形成する」
\end{description}

%\subsubsection*{以下}

\subsection{以降}

\begin{description}
\item[修正前:]「550\,nm以降の範囲では」
\item[修正後:]「550\,nm以上の範囲では」
\item[修正後:]「550\,nmより長波長の範囲では」
\item[修正後:]「550\,nmより低いエネルギーの範囲では」
\end{description}

「以降」が以上を指しているのか以下を指しているのか読者には判別できません。また光を波長で考えているのかエネルギーで考えているのか区別しにくい文脈では、区別しやすい表現に改めましょう。

\subsection{\% と\% ポイント}

\begin{description}
\item[修正前:]「2つの反射率測定では3\,\%の差が生じた」
\item[修正後:]「2つの反射率測定では3\,\%ポイントの差が生じた」
\item[修正後:]「2つの反射率測定では相対的に3\,\%の差が生じた」
\end{description}

比較する 2 つの値がそもそも何かの割合の場合、例えば2つの反射率測定値95\,\%と98\,\%を比較したい場合、これらの差を「XX\,\%の差」のように表現してしまうと読者には $|98-95|$ を指しているのか、$((\frac{98}{95} - 1)\times100$を指しているのか区別がつきません。前者を意図している場合には「\%ポイント」と書くことで、また後者の場合は「相対的に」を入れることによって、2つの意図を読者が区別できるようになります。

%\subsection{ピーク}

\section{冗長な表現}

\subsection{〇〇することができる}

\begin{description}
\item[修正前:]「抵抗と電流の積を求めることで電圧降下を推定することができる」
\item[修正後:]「抵抗と電流の積を求めることで電圧降下を推定できる」
\end{description}

少し格式ばった論文調の文章を書こうとすると頻出する表現です。もちろん日本語として間違いではありませんが、いたずらに文章を長くするだけで、読者に負担をかけます。

\subsection{〇〇をすることが可能である}

\begin{description}
\item[修正前:]「暗黒物質を検出することが可能である」
\item[修正後:]「暗黒物質を検出できる」
\end{description}

前節と同様です。

\subsection{〇〇を行った}

\begin{description}
\item[修正前:]「出力電圧のベースラインの補正を行った」
\item[修正後:]「出力電圧のベースラインを補正した」
\item[修正後:]「出力電圧のベースライン補正をした」
\end{description}

これは間違いではありませんが、「〇〇を行った」という表現が連続すると冗長になり読者にも負担をかけます。サ変動詞に書き換えるように意識しましょう。

%\subsection{となっている}

\subsection{〇〇の値}

\begin{description}
\item[修正前:]「電圧の値を測定した」
\item[修正後:]「電圧を測定した」
\end{description}

特に解説は要らないと思いますが、「の値」は冗長です。また測定対象は電圧であり、その結果を値として得るということに注意しましょう。

\subsection{しかしながら}

「しかし」で十分です。

\section{許容できない新語}

文語体の日本語文章ではまだ市民権を得ていませんが、口語で「ら」抜き言葉や「さ」入れ言葉を耳にする頻度がかなり高くなってきたため、修士論文などでも気にせずに使用する世代が出てくることが考えられます。本書の読者でもあまり意識したことがない人が多いと思われるため、予防線を張っておきます。

\subsection{「ら」抜き言葉}

「食べ{\bf ら}れる」を「食べれる」と書いたり、「見{\bf ら}れる」を「見れる」と書いたり、動詞の可能形で本来は入れる必要のある「ら」を抜いてしまう用法を「ら」抜き言葉と言います。これは五段活用をする動詞以外で頻繁に現れます。

\begin{description}
\item[修正前:]「チェンバー内部の様子を見れるよう石英窓を取り付けた」
\item[修正後:]「チェンバー内部の様子を見{\bf ら}れるよう石英窓を取り付けた」
\end{description}

\begin{description}
\item[修正前:]「エネルギー閾値を下げれないため」
\item[修正後:]「エネルギー閾値を下げ{\bf ら}れないため」
\end{description}

どのような場合が「ら」抜きなのかよく分からない人は、動詞の基本形を非定形にしてみましょう。「食べ{\bf る}」→「食べ{\bf ない}」、「見{\bf る}」→「見{\bf ない}」、「下げ{\bf る}」→「下げ{\bf ない}」のように基本形の最後が取り除かれて活用する動詞は、「食べ{\bf られる}」、「見{\bf られる}」」、「下げ{\bf られる}」のように活用し「ら」が入ります。

また「測{\bf る}」→「測{\bf らない}」、「読{\bf む}」→「読{\bf まない}」のように基本形の最後が他の母音になる(五段活用する)場合は「測{\bf れる}」、「読{\bf める}」のように活用し「ら」は入りません。

\subsection{「さ」入れ言葉}

「あり得なそう」を「あり得なさそう」のように書くのが「さ」入れ言葉です。これは「ない」という形容詞と「〜ない」という助動詞の活用を混同するために起きます。前者は「なさそう」と活用しますが、後者は「なそう」となります。

\begin{description}
\item[修正前:]「観測期間中に起きな{\bf さ}そうな事象はモンテカルロシミュレーションで考慮しなかった」
\item[修正後:]「観測期間中に起きなそうな事象はモンテカルロシミュレーションで考慮しなかった」
\end{description}

\subsection{ほぼほぼ}

「ほぼ」で十分です。

\section{役物の役割}

\subsection{鉤括弧「」}

鉤括弧は①発言や記述を引用する場合、②言葉を強調する場合、③本来の意味とは異なる意味を持たせる場合(もしくは少し皮肉を込める場合\footnote{英語ではこの用法で ``'' を使用します。これを知らずに間違えて使うと皮肉として受け取られますので注意してください。})、④読者に馴染みのない新しい言葉を紹介する場合などに使われます。物理系の修士論文ではほぼ使用する必要がないと思ってください。

\begin{description}
\item[③の例:]これまで「安全」とされてきた自動車の検査結果は全て不正によるものであった
\item[④の例:]これを「宇宙の晴れ上がり」と呼ぶ
\end{description}

\subsection{二重鉤括弧『』}

二重鉤括弧は鉤括弧中でさらに鉤括弧を使う場合や、書籍名を書くときに使われます。物理系の修士論文では、日本語の引用文献の書籍名を書く場面でしか使うことはありません。「」と『』は好みのほうを使って良いわけではありませんので、用法の違いに注意してください。

\subsection{中黒・}

中黒は似たようなものを一括りにして列挙するときなどに使用します。また漢字の連続を防ぎ可読性を高める効果もあります。他にも用途はありますが、割愛します。

\begin{description}
\item[修正前:]「チェレンコフ望遠鏡アレイでは大中小口径の異なる望遠鏡を設置し」
\item[修正前:]「チェレンコフ望遠鏡アレイでは大、中、小口径の異なる望遠鏡を設置し」
\item[修正後:]「チェレンコフ望遠鏡アレイでは大・中・小口径の異なる望遠鏡を設置し」
\end{description}

\begin{description}
\item[修正前:]「陽子陽子衝突」
\item[修正後:]「陽子・陽子衝突」
\end{description}

\begin{description}
\item[修正前:]「電子陽電子の対生成」
\item[修正後:]「電子・陽電子の対生成」
\end{description}

\subsection{エンダッシュ--}

エンダッシュ(\LaTeX ではハイフン2つを連続して入力する)は日本語にはない記号ですが、英語では①数値の範囲を表すとき、②2つの区間や事物を表すとき、③人名を繋げるときなどに使用します。

日本語では①と②は「〜」と「・」で置き換え可能ですので基本的に使う必要はありませんが、③は使うこともあるでしょう。またエンダッシュと間違えてハイフンを使ってはいけません。

\begin{description}
\item[修正前:]「1-100\,TeVのガンマ線」
\item[修正後:]「1--100\,TeVのガンマ線」(※英語でも同様)
\item[修正後:]「1〜100\,TeVのガンマ線」
\end{description}

\begin{description}
\item[修正前:]「陽子-陽子の衝突」
\item[修正後:]「陽子--陽子の衝突」(※英語でも同様)
\item[修正後:]「陽子・陽子の衝突」
\end{description}

\begin{description}
\item[③の例:]Schwarzschild--Couder光学系
\end{description}

\section{日本語で書けるカタカナ語}

日本語の日常会話では多数の外来語が使われており、その全てを簡単かつ一般的な日本語に直すことはできません。例えば「デジタル」という言葉を日本語に置き換えるのは困難です。

一方で、簡単な日本語で書けるにも関わらず、指導教員が研究指導中にカタカナ語を濫用するために影響を受けてしまう学生が数多くいます。物理の教員は英語を使うことが多いため咄嗟に対応する日本語が出てこないからですが、日本語の修士論文では可能な限り一般的な日本語を使うようにしましょう。

表~\ref{katakana}に、これまでの添削で頻繁に見られたカタカナ語と対応する日本語を示します。この表のうち、カタカナ語がすでに市民権を持っている場合には無理やり日本語に直す必要はありませんが、カタカナを乱用する学生はその語のを意味を日本語で説明できない場合が多々あります。そのような学生は積極的に日本語への言い換えをする訓練を意図的にしておくと良いでしょう。

\begin{table}
  \centering
  \caption{平易な日本語で置き換え可能なカタカナ語。絶対に全て日本語にせよという意味ではありませんので、状況に応じて使い分けてください。}
  \begin{tabular}{lll}
    \hline
    カタカナ語 & 日本語 & 使用例 \\
    \hline
    コスト(cost) & 費用 & 建設コスト → 建設費用 \\
    イメージ(image) & 模式図、概念図、想像図 & 測定系のイメージ → 測定系の概念図 \\
    システム(system) & 系 & 測定システム → 測定系 \\
    セットアップ(setup) & 配置、構成 & 測定のセットアップ → 測定系の構成 \\
    カウント(count) & 計数 & 検出光子数をカウントした → 検出光子数を計数した \\
    ベースライン(baseline) & 基準 & ベースライン電圧 → 基準電圧 \\
    アノード(anode) & 陽極 & アノード電圧 → 陽極電圧 \\
    カソード(cathode) & 陰極 & カソード電流 → 陰極電流 \\
    ダークカレント(dark current) & 暗電流 & \\
    ディフューズ(diffuse) & 拡散 & \\
    アンプ(amplifier) & 増幅器 & \\
    イベント(event) & 事象 & \\
    サンプル(sample) & 個体 & サンプルごとの差 → 個体差 \\
    サンプル(sample) & 標本 & \\
    ゲイン(gain) & 利得、増倍率、増幅率 & \\
    サイズ(size) & 大きさ、面積、長さ & \\
    チェンバー(chamber) & 槽 & 真空チェンバー → 真空槽 \\
    ノイズ(noise) & 雑音 &  \\
    バックグラウンド(background) & 背景 & バックグラウンドイベント → 背景事象 \\
    シグナル(signal) & 信号 &  \\
    ピーク(peak) & 最大値、最頻値 &  \\
    プロット(plot) & 図(種類に応じて適切な対応語を使う) &  \\
    パラメータ(parameter) & 変数、性能 &  \\
    コインシデンス(coincidence) & 同時発生、同時計数 &  \\
    ターゲット(target) & 標的 & ターゲット粒子 → 標的粒子 \\
    スケール(scale) & 級、規模 & トンスケールの実験 → トン規模の実験 \\
    \hline
  \end{tabular}
  \label{katakana}
\end{table}

また、英単語の本来の意味を理解していないにも関わらず不用意にカタカナ語を使うことで、間違った言葉の使い方をする事例もよく見かけます。表~\ref{wrong_katakana}にいくつかまとめます。

\begin{table}
  \centering
  \caption{英単語本来の意味を知らずに誤用されているのを見かけるカタカナ語。}
  \begin{tabular}{lll}
    \hline
    カタカナ語 & 日本語 & 誤用例 \\
    \hline
    イメージ(image) & 模式図、概念図、想像図 & 測定系のイメージ → 測定系の概念図 \\
    パターン(pattern) & 型、規範、繰り返し模様など & ハドロンシャワーのパターン → ハドロンシャワーの像 \\
    ピーク(peak) & 最大値、最頻値(本来の意味は山頂) & 裾の広いピーク → 裾の広い分布 \\
    ボトルネック(bottle neck) & 律速する物 & 暗電流の大きさがSiPMのボトルネックになる → 暗電流の大きさがSiPMの弱点になる \\
    \hline
  \end{tabular}
  \label{wrong_katakana}
\end{table}

