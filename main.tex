\documentclass[twoside,openright,uplatex]{jsbook}
% uplatex オプションを指定し、ユニコード対応に。ただだし、uplatex でコンパイルすること。

% jsbookで余白が広すぎるのを直す
% 参照 https://oku.edu.mie-u.ac.jp/~okumura/jsclasses/
\setlength{\textwidth}{\fullwidth}
\setlength{\evensidemargin}{\oddsidemargin}

% 同梱の ISEE 用の表紙テンプレ
\usepackage{thesis_cover}

% OTF フォントを使えるようにし、複数のウェイトも使用可能にする。
% これがないと、Mac のヒラギノ環境で使われる角ゴが太すぎてみっともない。
\usepackage[deluxe]{otf}

% OT1→T1に変更し、ウムラウトなどを PDF 出力で合成文字ではなくす
\usepackage[T1]{fontenc}

% uplatex の場合に必要な処理 
\usepackage[utf8]{inputenc} % エンコーディングが UTF8 であることを明示する。
\usepackage[prefernoncjk]{pxcjkcat} % アクセントつきラテン文字を欧文扱いにする

% Helvetica と Times を sf と rm のそれぞれで使う。
% default だとバランスが悪いので、日本語に合わせて文字の大きさを調整する。
\usepackage[scaled=1.05,helvratio=0.95]{newtxtext}

% 画像の取り扱いに必要
\usepackage[dvipdfmx]{graphicx}

% 表でセルを複数列で結合する
\usepackage{multicol}

% 数式の機能を拡張
\usepackage{amsmath}

% citep や citet を有効にする
\usepackage{natbib}

% (Okumura, 2009) などを (Okumura 2009) とする
\setcitestyle{aysep={}}

% subfigure 環境で、(a)、(b) などの番号を左上に表示する。宇宙系の分野ではこれが一般的なはず。
\usepackage[nooneline]{subfigure}
\subfiguretopcaptrue

% 行番号を表示する。添削時のみに使い、事務提出版ではコメントアウトする
%\usepackage{lineno}
%\linenumbers

% PDF 内で外部リンクや文書内リンクを生成したい場合に使う(好みによる)
% \usepackage[dvipdfmx]{hyperref}

% newcommand を使うことで、繰り返し使う長ったらしい入力を簡単にすることができる
\makeatletter
\newcommand{\ion}[2]{#1$\;${\small\rmfamily\@Roman{#2}}\relax}%
\makeatother
\newcommand{\HI}{\mbox{\ion{H}{1}}} % 中性原子ガス(HI 領域)の例
\newcommand{\bs}{\symbol{92}} % backslash

% 氏名などの情報が入っているファイル。各自で編集。
\seireki{2019} % 年度
\articletype{卒業論文} % 論文のタイプ(卒業論文 or 修士論文)
\title{添削者を困らせることのない\\修士論文の書き方の研究} % 論文題目


\labname{宇宙線物理学研究室} %研究室名
% 下記いずれか
\departmentname{工学部土木工学科} % 所属学科
% \departmentname{理工学研究科\\社会基盤学専攻} % 所属専攻


\StudentGrade{} % 所属課程(学部生は空欄,修士は「博士課程(前期課程)2年」など)
\StudentIdNumber{123456} % 学籍番号
\author{奥村 曉} % 氏名

\begin{document}

\frontmatter

\maketitle

ここには論文の概要(abstract)を書きます。論文の先頭なので早い時期に書き始める人がいますが、論文の結論や論理展開はなかなか執筆終盤まで固まりません。そのため、論文の流れや結論がかなり明確になった最終段階で書くようにしましょう。

概要は論文全体の内容を短文で説明するものですので、研究の背景と目的、研究内容、結果と結論などが全て網羅されている必要があります。ここを読んだだけで、論文の中身が大雑把に把握できるようにすることが大切です。原則として改行せずに1段落で書きますが、これは複数段落に分けて書くような文章を無理やり1段落に合体させろということではありません。1段落で流れるように書いてください。文量としてはA4の半分から3分の2程度だと思います。2ページにもわたる概要はありえません。

この概要と序論を同一視する人が多く見られます。おそらく修論提出の直前になって慌てて書くため、序論を要約する格好になってしまうのでしょう。しかし序論には研究結果の詳細や、それに対しての議論や考察は書かれていないことが一般的です。そのため序論の焼き直しのような概要を書かれてしまうと、論文の結論や議論を概要で追うことができなくなり概要を書く意味が薄れてしまいますので、「論文全体の内容を短くまとめているか」に注意して要約してください。


\tableofcontents
\listoffigures
\listoftables

\mainmatter

% include を使うことで、別ファイルに分割することができます。
\chapter{序論}
\label{chap_intro}

英語で言うところのイントロダクションです。通常、「序論」(introduction)で始める場合は「結論」(conclusion)という章で締めます。もし「はじめに」で始まる場合は「おわりに」です。

この章では研究の背景や課題などを簡潔に説明します。2から4ページもあれば十分ですし、細かく節に分ける必要もありません。この章で必要なことは、なぜこの論文が書かれたのか、過去の研究に対する位置付け・課題は何か、この研究でどこまでを明らかにしようとしているのかを少ないページ数で説明することです。

このような序論の存在しない修士論文はたくさん存在しますが、何十ページにもなる修士論文では研究の位置付けや課題がどこに書かれているのか読者は見失いやすくなります。先頭に独立した章で簡潔に道筋を示すことで、続く章を読者が読みやすくなります。

\chapter{ガンマ線天文学とCTA計画}
\label{chap_review}

この章では、自分の研究に関連する分野の歴史や現状について説明したり、研究を展開する上で重要となる知識の解説を行います。ここで使用している見出し「ガンマ線天文学…」はあくまで例ですが、もしCherekov Telescope Array(CTA)計画\footnote{省略語は必ず正式名称を先に書き、省略系は丸括弧に入れます。省略語はあくまで「以降このように略す」という用途だからです。また、日本語文章中で使う丸括弧は()ではなく()です。}に携わる院生の書く修士論文であれば、ガンマ線天文学や宇宙線物理学全般について、現行望遠鏡とガンマ線観測の原理について、またCTA計画についての記述がこの章では期待されます。

場合によっては「序論」と合体させても良いですが、本章は比較的長くなり結論に直結しない情報もたくさん出てくるため、独立した章である方が読者は読みやすいでしょう。

またこの章が長くなるときには、例えば「ガンマ線天文学」と「CTA計画」のように、2つの章に分割するというのも良いと思います。

\chapter{自分の研究本体を述べるところ}
ここは自分のやった研究を述べる章です。実際の中身に合わせて章を複数立てにする場合もあると思います。「議論」の章を別に分ける場合は、この章では得られた結果までを記述し、その結果に対する議論は「議論」の章に回すのが良いでしょう。この章は必ずしも1つの章のみである必要はありません。研究内容に応じて、複数の章に分割することも一般的に行われます。

修士論文で大切なことは、第~\ref{chap_intro}~章や第~\ref{chap_review}~章で述べた伏線(研究の目的と動機)を回収するべく、きちんと研究内容を順序立てて書き、また自分の貢献を明確にすることです。論文全体で論理展開がきちんとしていれば良いので、必ずしも実際に行った実験などの時系列でこの章を書き進める必要はありません。また修士論文としての完成度が大切ですので、修士論文のテーマに直接関係のない自分のやったことを無理に混ぜる必要もありません。

\chapter{議論}
ここではこの研究で得られた結果についての議論を行います。測定結果や観測結果などと一緒に議論を進める場合もあるので、必ず必要な章であるとは言えませんが、できる限り研究で得られた事実と自分の議論は分けましょう。

\chapter{結論}
ここには自分の修士論文の結論を書きます。「議論」の章で書かれたことも、再びここに短く書かれます。

「序論」で始めたら「結論」、「はじめに」で始めたら「おわりに」が原則です。ただし、「まとめと今後の展望」などとすることもありますので、好みに応じて変えてください。

\chapter*{付録} % 章番号を出さない
\addcontentsline{toc}{chapter}{付録} % 目次に載せる

「付録」(appendix)は、論文の本文に載せるには情報として邪魔もしくは必須ではないものの、読者にとって有益となるような情報を載せます。付録を必要としない論文ももちろん存在しますので、そこは著者の判断です。

例えば、たくさんの観測データを様々なモデルでフィットした場合、フィット結果の絵がたくさん出てくるはずです。そのような図は本文中に大量に出されても大切な情報を見失ってしまいますので、大部分は付録に載せることが推奨されます。他には、何かしらの長い式変形や証明を載せる必要がある場合、付録に移動する場合があります。

% 付録は chapter の 1 つとして作りますが、章番号は表示しません。
% また付録の 1 つずつはアルファベットで番号付けをするのが一般的です。
\setcounter{section}{0} % section の番号をゼロにリセットする
\renewcommand{\thesection}{\Alph{section}} % 数字ではなくアルファベットで数える
\setcounter{equation}{0} % 式番号を A.1 のようにする
\renewcommand{\theequation}{\Alph{section}.\arabic{equation}}
\setcounter{figure}{0} % 図番号
\renewcommand{\thefigure}{\Alph{section}.\arabic{figure}}
\setcounter{table}{0} % 表番号
\renewcommand{\thetable}{\Alph{section}.\arabic{table}}

\section{すごい長い証明}
式~(\ref{eq})のように、式番号がアルファベットとアラビア数字の組み合わせになるように、\LaTeX{}ソース中で設定してありますので、中身を眺めてみてください。

\begin{equation}
  \label{eq}
  1 + 1 = 2
\end{equation}


\section{すごいたくさんのフィットの図}

\section{修士論文添削前に自己点検する項目}

\begin{itemize}
\item[\CID{00728}] 第\ref{chap:plagiarism}章を読み、剽窃について十分に理解したか。
\item[\CID{00728}] 修士論文に剽窃箇所もしくは剽窃と見なされうる箇所は存在しないか。
\item[\CID{00728}] \LaTeX\ で図番号などの参照先がないせいで「図??」「表??」「??節」のようになっている箇所はないか。
\item[\CID{00728}] 日本語読点「、」と欧文カンマ「,」が混在していないか。例えば「ガンマ線望遠鏡は、HESS, MAGIC, VERITASなどがある」。
\item[\CID{00728}] 日本語丸括弧「()」と欧文丸括弧「()」が混在していないか。
\item[\CID{00728}] 単位と数値の間にスペースは入っているか。「100MeV」など。
\item[\CID{00728}] 単位が斜体になっていないか。「$100~MeV$」など。
\item[\CID{00728}] 変数でない添字などが斜体になっていないか。$N_{trigger}$など。
\item[\CID{00728}] 自分で作成したものではない図や写真は、全て出典が明記され、転載であることを書いてあるか。
\end{itemize}

\chapter*{謝辞}%
\addcontentsline{toc}{chapter}{謝辞}

一般的に論文における「謝辞」(acknowledgments)とは、その論文を作成する上で不可欠だった様々な支援に対する感謝の気持ちを述べる場所です。通常の投稿論文であれば、その論文の作成者・共同研究者は主著者(筆頭著者もしくは責任著者)や共著者として著者リストに入っています。しかし修士論文は単著で書くため、例えば指導教員から論文のアイデアをもらっても指導教員は共著者になりません。また様々な共同研究者にデータ解析を手伝ってもらったり、実験データを提供してもらった場合にも、これが普通の論文であれば共著者になりうるところですが、修士論文では単著、つまりあなたの名前だけが記載されます。そこで、謝辞の必要性が生じるのです\footnote{複数著者による投稿論文の場合、共著者にするべきか謝辞での言及のみに留めるかは、場合によります。}。

先輩の修論の謝辞を真似て、やたらと大人数への謝辞を並べてある修論をよく見かけます。私の所属する研究室は大所帯のため、教員全員、院生全員、事務補佐員が20人以上並んでいるものがあります。もちろん感謝を述べたかったら書くのは構いませんしそれを止めるつもりはありませんが、修論の読者からすると「別に大して感謝の気持ちのないくせに、取捨選択する度胸がなく、差し障りのないように機械的に羅列しただけだろう」という印象を持ち、逆に謝辞としての意味が薄れます。

絶対に謝辞に含めなくてはいけないのは、およそ次の通りです\footnote{これは投稿論文にも当てはまりますが、投稿論文では、貢献の大きい人は共著者になるべきです。}。順序は前後しても構いませんが、貢献度の高い人を前にするべきです。
\begin{enumerate}
\item 指導教員(教授、准教授など)
\item 修士論文の研究テーマやアイデアを考えた人、提供した人(指導教員の場合が多い)
\item 指導教員以外で直接的に指導した人(助教、ポスドクなど)
\item 論文において本質的となる議論や指摘を行ってくれた人
\item 実験やデータ解析をかなり手伝ってくれた人(先輩や共同研究者など)
\item データや解析スクリプトなどを提供してくれた人(共同研究者など)
\item 研究資金を受給していれば、その資金名と提供元(学内の研究支援も含む)
\end{enumerate}

修士論文の場合、さらに同期の学生や事務補佐員などを謝辞に加える学生も多くいます。これは科学論文として必須ではありませんので、謝辞に加えないからといって失礼に当たるわけではありません。また家族への謝辞を加えることもよくあります\footnote{独立生計でない場合、研究資金の提供先ですので当然と言えば当然かもしれません。}。

また謝辞では次のことに注意してください。

\begin{enumerate}
\item 現在は指導教「員」と呼称し、指導教「官」ではない\footnote{国立大学法人化によって、大学教員は公務員ではなくなったため。}。
\item 職階を間違えないこと。「准教授」を「助教授」、「助教」を「助手」とする間違いが多い\footnote{2007年に名称が変更となった。}。
\item 氏名の漢字を絶対に間違えない。旧字体(曉と暁など)や異体字(齋藤と斎藤など)に気をつけること。
\item 所属先は略称ではなく正式名称を書くこと。大学の研究所などの場合、研究所名だけでなく大学名を先に書くこと。
\item 敬称をつけること(「2年間にわたり御指導くださった XX 教授」など)。博士号を持っている人には「博士」をつけること。
\end{enumerate}


\renewcommand{\bibname}{引用文献}
\bibliographystyle{jecon}
\bibliography{thesis}

\end{document}
