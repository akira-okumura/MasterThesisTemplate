\RequirePackage{plautopatch}
\documentclass[twoside,openright,a4paper,papersize,dvipdfmx]{jsbook}
\usepackage[]{multirow}
\usepackage{graphicx}
\usepackage{siunitx}
% easy-todoは目次の体裁設定と競合するため使用不可
\usepackage{todonotes}

% 修論本体と表紙で共通で必要となる設定
% jsbookで余白が広すぎるのを直す
% 参照 https://web.archive.org/web/20230119032705/https://oku.edu.mie-u.ac.jp/~okumura/jsclasses/
% https://github.com/texjporg/jsclasses
\setlength{\textwidth}{\fullwidth}
\setlength{\evensidemargin}{\oddsidemargin}
\addtolength{\textwidth}{-5truemm}
\addtolength{\oddsidemargin}{5truemm}

% 同梱の ISEE 用の表紙テンプレ
\usepackage{thesis_cover}

% 色
\usepackage[dvipdfmx]{color}

% OTF フォントを使えるようにし、複数のウェイトも使用可能にする。
% これがないと、Mac のヒラギノ環境で使われる角ゴが太すぎてみっともない。
\usepackage[deluxe]{otf}

% OT1→T1に変更し、ウムラウトなどを PDF 出力で合成文字ではなくす
\usepackage[T1]{fontenc}

% uplatex の場合に必要な処理 
\usepackage[utf8]{inputenc} % エンコーディングが UTF8 であることを明示する。
\usepackage[prefernoncjk]{pxcjkcat} % アクセントつきラテン文字を欧文扱いにする

% Helvetica と Times を sf と rm のそれぞれで使う。
% default だとバランスが悪いので、日本語に合わせて文字の大きさを調整する。
\usepackage[scaled=1.05,helvratio=0.95]{newtxtext}

% latexdiff
% 実際の修論には入れる必要なし
%DIF PREAMBLE EXTENSION ADDED BY LATEXDIFF
%DIF UNDERLINE PREAMBLE %DIF PREAMBLE
\RequirePackage[normalem]{ulem} %DIF PREAMBLE
\RequirePackage{color}\definecolor{RED}{rgb}{1,0,0}\definecolor{BLUE}{rgb}{0,0,1} %DIF PREAMBLE
\providecommand{\MyDIFadd}[1]{{\protect\color{blue}\uwave{#1}}} %DIF PREAMBLE
\providecommand{\MyDIFdel}[1]{{\protect\color{red}\sout{#1}}}                      %DIF PREAMBLE
%DIF SAFE PREAMBLE %DIF PREAMBLE
\providecommand{\MyDIFaddbegin}{} %DIF PREAMBLE
\providecommand{\MyDIFaddend}{} %DIF PREAMBLE
\providecommand{\MyDIFdelbegin}{} %DIF PREAMBLE
\providecommand{\MyDIFdelend}{} %DIF PREAMBLE
%DIF FLOATSAFE PREAMBLE %DIF PREAMBLE
\providecommand{\MyDIFaddFL}[1]{\MyDIFadd{#1}} %DIF PREAMBLE
\providecommand{\MyDIFdelFL}[1]{\MyDIFdel{#1}} %DIF PREAMBLE
\providecommand{\MyDIFaddbeginFL}{} %DIF PREAMBLE
\providecommand{\MyDIFaddendFL}{} %DIF PREAMBLE
\providecommand{\MyDIFdelbeginFL}{} %DIF PREAMBLE
\providecommand{\MyDIFdelendFL}{} %DIF PREAMBLE
%DIF END PREAMBLE EXTENSION ADDED BY LATEXDIFF


% 表でセルを複数列で結合する
\usepackage{multicol}

% 数式の機能を拡張
\usepackage{amsmath}

% citep や citet を有効にする
\usepackage{natbib}

% (Okumura, 2009) などを (Okumura 2009) とする
% 日本語文章で全角丸括弧の表示にし、かつ \inhibitglue で役物同士の字間を適切にする。
% https://oku.edu.mie-u.ac.jp/tex/mod/forum/discuss.php?d=2349
\setcitestyle{aysep={},notesep={},open={\inhibitglue(},close={)\inhibitglue}}

% bibliography を目次に追加
\usepackage[nottoc,notlot,notlof]{tocbibind}

% 行番号を表示する。添削時のみに使用し,提出時はコメントアウト
%\usepackage{lineno}
%\linenumbers

% subfigure 環境で、(a)、(b) などの番号を左上に表示する。宇宙系の分野ではこれが一般的なはず。
\usepackage[nooneline]{subfigure}
\subfiguretopcaptrue

% PDF 内で外部リンクや文書内リンクを生成したい場合に使う(好みによる)
% 印刷時に色が出るかどうかは、使用する PDF viewer の挙動による。
% 紙媒体で修論を提出する場合、文字色は黒にするのが適切なので要注意。
\usepackage[colorlinks=true,allcolors=blue]{hyperref}
% 色を個別に変更したい場合の例(あまり勧めない)
%\hypersetup{
%    colorlinks=true,
%    citecolor=red,
%    linkcolor=blue,
%    urlcolor=green,
%}

% newcommand を使うことで、繰り返し使う長ったらしい入力を簡単にすることができる
\makeatletter
\newcommand{\ion}[2]{#1$\;${\small\rmfamily\@Roman{#2}}\relax}%
\makeatother
\newcommand{\HI}{\mbox{\ion{H}{1}}} % 中性原子ガス(HI 領域)の例
\newcommand{\bs}{\symbol{92}} % backslash
\newcommand{\red}[1]{\textcolor{red}{#1}}
\newcommand{\ured}[1]{\textcolor{red}{\underline{\textcolor{black}{#1}}}}
\newcommand{\ugreen}[1]{\textcolor{green}{\underline{\textcolor{black}{#1}}}}
\newcommand{\ublue}[1]{\textcolor{blue}{\underline{\textcolor{black}{#1}}}}

% %%%%%%%%%%%%%%%%%%%%%%%%%%%%%%%%%%%%%%%%%%%%%%%%%%%%%%%%%%%%%%%%%%%%%%%%%%%%%%%%%%%%%%%%%%%
% %%%%%%%%%%%%%%%%%%%%%%%%%%%%%%%%%%%% ここから本体 %%%%%%%%%%%%%%%%%%%%%%%%%%%%%%%%%%%%%%%
% %%%%%%%%%%%%%%%%%%%%%%%%%%%%%%%%%%%%%%%%%%%%%%%%%%%%%%%%%%%%%%%%%%%%%%%%%%%%%%%%%%%%%%%%%%%

\begin{document}

\frontmatter

% 氏名などの情報が入っているファイル。各自で編集。
\seireki{2019} % 年度
\articletype{卒業論文} % 論文のタイプ(卒業論文 or 修士論文)
\title{添削者を困らせることのない\\修士論文の書き方の研究} % 論文題目


\labname{宇宙線物理学研究室} %研究室名
% 下記いずれか
\departmentname{工学部土木工学科} % 所属学科
% \departmentname{理工学研究科\\社会基盤学専攻} % 所属専攻


\StudentGrade{} % 所属課程(学部生は空欄,修士は「博士課程(前期課程)2年」など)
\StudentIdNumber{123456} % 学籍番号
\author{奥村 曉} % 氏名

% 表紙
\maketitle

% これを入れることでページ番号が表示されない。
\thispagestyle{empty}

% 概要
ここには論文の概要(abstract)を書きます。論文の先頭なので早い時期に書き始める人がいますが、論文の結論や論理展開はなかなか執筆終盤まで固まりません。そのため、論文の流れや結論がかなり明確になった最終段階で書くようにしましょう。

概要は論文全体の内容を短文で説明するものですので、研究の背景と目的、研究内容、結果と結論などが全て網羅されている必要があります。ここを読んだだけで、論文の中身が大雑把に把握できるようにすることが大切です。原則として改行せずに1段落で書きますが、これは複数段落に分けて書くような文章を無理やり1段落に合体させろということではありません。1段落で流れるように書いてください。


この文書の\LaTeX {}ファイルは、\url{https://github.com/akira-okumura/MasterThesisTemplate}から入手可能です。

% TODO list (提出時にコメントアウト)
\listoftodos

% 目次
\tableofcontents
\listoffigures
\listoftables

% ここから内容に入っていく
\mainmatter

% include を使うことで、別ファイルに分割することができます。
\chapter{序論}
\label{chap_intro}

英語で言うところのイントロダクションです。通常、「序論」(introduction)で始める場合は「結論」(conclusion)という章で締めます。もし「はじめに」で始まる場合は「おわりに」です。

この章では研究の背景や課題などを簡潔に説明します。2から4ページもあれば十分ですし、細かく節に分ける必要もありません。この章で必要なことは、なぜこの論文が書かれたのか、過去の研究に対する位置付け・課題は何か、この研究でどこまでを明らかにしようとしているのかを少ないページ数で説明することです。

このような序論の存在しない修士論文はたくさん存在しますが、何十ページにもなる修士論文では研究の位置付けや課題がどこに書かれているのか読者は見失いやすくなります。先頭に独立した章で簡潔に道筋を示すことで、続く章を読者が読みやすくなります。

\chapter{ガンマ線天文学とCTA計画}
\label{chap_review}

この章では、自分の研究に関連する分野の歴史や現状について説明したり、研究を展開する上で重要となる知識の解説を行います。ここで使用している見出し「ガンマ線天文学…」はあくまで例ですが、もしCherekov Telescope Array(CTA)計画\footnote{省略語は必ず正式名称を先に書き、省略系は丸括弧に入れます。省略語はあくまで「以降このように略す」という用途だからです。また、日本語文章中で使う丸括弧は()ではなく()です。}に携わる院生の書く修士論文であれば、ガンマ線天文学や宇宙線物理学全般について、現行望遠鏡とガンマ線観測の原理について、またCTA計画についての記述がこの章では期待されます。

場合によっては「序論」と合体させても良いですが、本章は比較的長くなり結論に直結しない情報もたくさん出てくるため、独立した章である方が読者は読みやすいでしょう。

またこの章が長くなるときには、例えば「ガンマ線天文学」と「CTA計画」のように、2つの章に分割するというのも良いと思います。

\chapter{自分の研究本体を述べるところ}
ここは自分のやった研究を述べる章です。実際の中身に合わせて章を複数立てにする場合もあると思います。「議論」の章を別に分ける場合は、この章では得られた結果までを記述し、その結果に対する議論は「議論」の章に回すのが良いでしょう。この章は必ずしも1つの章のみである必要はありません。研究内容に応じて、複数の章に分割することも一般的に行われます。

修士論文で大切なことは、第~\ref{chap_intro}~章や第~\ref{chap_review}~章で述べた伏線(研究の目的と動機)を回収するべく、きちんと研究内容を順序立てて書き、また自分の貢献を明確にすることです。論文全体で論理展開がきちんとしていれば良いので、必ずしも実際に行った実験などの時系列でこの章を書き進める必要はありません。また修士論文としての完成度が大切ですので、修士論文のテーマに直接関係のない自分のやったことを無理に混ぜる必要もありません。

\chapter{\LaTeX{}の使い方}
本章では、\LaTeX{}の使い方を以下説明します。\textbf{\textsf{ここでの表示例は本PDFを読むだけではどのような\LaTeX{}コードに対応しているか分かりませんので、\texttt{main.tex}や\texttt{LaTeX.tex}の中身を参照してください。}}

このPDF文書中に\texttt{command}のような書体で記載されているものは、\LaTeX{}ソース中で実際に入力するコマンドやファイル名を示しています。

\section{節の使い方}
\texttt{\bs{}section}や\texttt{\bs{}subsection}を使うと「節」(section)と呼ばれる構造を作ることができます。長い章を分割して論理展開を分かりやすくする目的で使います。

文中で節を参照するときは、\texttt{section}であっても\texttt{subsection}であっても「節」と呼び、「\ref{sec_figure}~節」や「第\ref{sec_figure}~節」のように書きます(\texttt{ref}コマンドの使用は次節参照)。章を参照するときは「\ref{chap_review}~章」や「第\ref{chap_review}~章」とします。

\section{図の使い方}
\label{sec_figure} % このようにラベルをつけることで、\refコマンドで節や図の番号を参照できます。

論文中に図を入れるときは、\texttt{figure}環境を使用します。画像形式は図~\ref{fig_CTA}のようなJPEG(主に写真などに最適)やPNG(色数の少ない画像に最適)に加え、図~\ref{fig_histogram}のようにPDF(グラフなどに最適)も使うことができます。実際の使い方は、この\LaTeX{}のコードを読んでください。\textsf{\textbf{EPS形式はいまどき誰も使いません。古い\LaTeX{}の本や年寄りに騙されないでください。}}

\begin{figure} % 特に強い理由がない限り、[htbp]のような指定はしないでください。
  \centering
  % 図の横幅をちょうど良い具合に自分で調整します。
  % 図中の文字を読めないような大きさにはしないでください。
  \includegraphics[width=14cm]{fig/CTA.jpg}
  % 図の説明が長い場合、[]で囲むことで短い図の説明を目次のみに表示できます。
  \caption[CTAの完成想像図]{CTAの完成想像図(画像提供:G.~Pérez、IAC、SMM)。JPEG(ビットマップ画像)なので、出力PDFで拡大するとドットが見えます。}
  % これで本文中から参照できます。
  \label{fig_CTA}
\end{figure}

\begin{figure}
  \centering
  % PDFも使えます。
  \includegraphics[width=14cm]{fig/histogram.pdf}
  \caption[ガウシアンでヒストグラムをフィットした例]{ガウシアンでヒストグラムをフィットした例。PDF(ベクター画像)なので、出力PDFで拡大しても滑らかです。また文字列もPDF中で検索することができます。}
  \label{fig_histogram}
\end{figure}

図を文中で参照したいときは\texttt{ref}コマンドを使用して、「図~\ref{fig_CTA}」のようにすることができます。この部分は\LaTeX{}中で実際には\texttt{図\~{}\bs{}ref\{fig\_CTA\}}と書いています。「図」と\texttt{\bs{}ref}の間に\texttt{\~{}}を入れるのは、「図」と図番号の間で改行を防ぐためです\footnote{このようにチルダを入れる手法は、人名の姓名の間や数値と単位の間で改行を防ぐのにも広く使われます。}。

\texttt{figure}環境で図を挿入する場所は、初めてその図を言及する段落の直後、もしくは直前です。あまりに離れた場所に図を挿入すると読者はどこに図があるかを探さなくてはならず、読むのが困難になるからです。

場合によっては複数の図を並べたいこともあるでしょう。そのようなときは、\texttt{subfigure}環境を使って図~\ref{fig_subfigure}のようにすることができます。\texttt{minipage}環境でも似たようなことができますが、\texttt{subfigure}を使うと小番号を自動で付与したり、「図~\ref{fig_subfigure_b}」のように、小番号を参照することができます。

\begin{figure}
  \centering
  \subfigure[]{% {の直後に%を置くことで、改行をさせない(図(b)を改行させない)。
    \includegraphics[width=.5\textwidth,clip]{fig/histogram.pdf}%
    \label{fig_subfigure_a}%
  }%
  \subfigure[]{%
    \includegraphics[width=.5\textwidth,clip]{fig/histogram.pdf}%
    \label{fig_subfigure_b}%
  }
  \subfigure[]{%
    \includegraphics[width=.5\textwidth,clip]{fig/histogram.pdf}%
    \label{fig_subfigure_c}%
  }%
  \subfigure[]{%
    \includegraphics[width=.5\textwidth,clip]{fig/histogram.pdf}%
    \label{fig_subfigure_d}%
  }
  \caption[複数の図を並べた例]{複数の図を並べた例。(a)ガウシアンフィット。(b)同じもの。(c)これも同じもの。(d)これも同じもの。}
\label{fig_subfigure}
\end{figure}

またせっかく図の並べ方が分かったので、同じ図をPDF、PNG、JPEGにして図~\ref{fig_formats}にて比較してみましょう。それぞれの画像の特徴が分かります。また図~\ref{fig_formats}は参考のため\texttt{subfigure}ではなく\texttt{minipage}環境を使って作ってあります。

\begin{figure}
  \begin{minipage}[b]{.3333\linewidth}
    \leftline{(a)}
    \centering
    \includegraphics[width=5.5cm]{fig/histogram.pdf}
  \end{minipage}%
  \begin{minipage}[b]{.3333\linewidth}
    \leftline{(b)}
    \centering
    \includegraphics[width=5.5cm]{fig/histogram.png}
  \end{minipage}%
  \begin{minipage}[b]{.3333\linewidth}
    \leftline{(c)}
    \centering
    \includegraphics[width=5.5cm]{fig/histogram.jpg}
  \end{minipage}
  \caption[異なる画像形式の比較]{異なる画像形式の比較。(a) PDF形式。拡大しても綺麗であり、文字も検索やコピーができる。(b) PNG形式。拡大するとビットマップ画像であることが分かる。文字を選択できない。(c) JPEG形式。PNGに比べ、JPEG圧縮特有のブロックノイズ、モスキートノイズが発生しており非常に汚いことが分かる。}
  \label{fig_formats}
\end{figure}

\section{表の使い方}

表~\ref{tab_cta}に、\LaTeX{}でどのように表を作成するかの例を示します。実際にどういう \LaTeX{}コードがこの表に対応するのかは、ファイルの中身を眺めてください。

\begin{table} % 表も[htbp]のような場所指定は特に必要ない
  \centering
  % 表のキャプションは必ずその表の上に来ます。図の場合は下です。違いに気をつけてください。
  \caption{CTA で使用される望遠鏡の性能諸元}
  \footnotesize % 横幅のある表なので、文字サイズを小さくしています。通常は必要ありません。
  \label{tab_cta} % ラベルのつけ方は図と同様です。
  \begin{tabular}{lccccccc} % 列が何列あるかを示します。lcrはそれぞれ左・中央・右揃えの指定です。
    \hline
    &
    \shortstack{大口径望遠鏡 \\ Large-Sized Telescope \\ (LST)} &
    % 複数列を結合したいときは、multicolumnを使います。
    \multicolumn{2}{c}{\shortstack{中口径望遠鏡 \\ Medium-Sized Telescope \\ (MST)}} &
    \shortstack{SC 中口径望遠鏡 \\ Schwarzschild--Couder MST \\ (SC-MST)} &
    \multicolumn{3}{c}{\shortstack{小口径望遠鏡 \\ Smalle-Sized Telescope \\ (SST)}} \\
    & & FlashCam & NectarCAM & & GCT & ASTRI & 1M-SST \\
    \hline
    エネルギー範囲 & 20--200 GeV & \multicolumn{2}{c}{100 GeV -- 10 TeV} & 200 GeV -- 10 TeV & \multicolumn{3}{c}{5--300 TeV} \\
台数(北半球)& 4 & \multicolumn{2}{c}{15} & 0 & \multicolumn{3}{c}{0} \\
台数(南半球)& 4 & \multicolumn{2}{c}{24} & 24 & \multicolumn{3}{c}{70--90} \\
鏡直径 &	23~m & \multicolumn{2}{c}{12~m} & 9.7~m & 4~m & 4~m & 4~m \\
焦点距離 & 28~m & \multicolumn{2}{c}{16~m} & 5.6~m & 2.3~m & 2.15~m & 5.6~m \\
視野 & 4.5$^\circ$ & \multicolumn{2}{c}{7.7$^\circ$} & 8$^\circ$ & 8.6$^\circ$ & 9.6$^\circ$ & 9$^\circ$ \\
光学系 & 放物鏡 & \multicolumn{2}{c}{Davies--Cotton (DC)} & Schwarzschild--Couder (SC) & SC & SC & DC \\
画素数 & 1,855 & 1,764 & 1,855 & 11,328 & 2,048 & 1,984 & 1,296\\
\hline
  \end{tabular}
  \normalsize % 文字サイズを元に戻します
\end{table}

論文中で使う表の一般的な注意点として、あまり罫線をたくさん使いすぎないことです。日本では全てのセルの周辺に罫線を使う傾向があり、最悪、表~\ref{tab_cta_bad}のようになります。窮屈になるので、このような罫線の多用はやめましょう。

\begin{table}
  \centering
  \caption{表~\ref{tab_cta}の悪い例}
  \footnotesize
  \label{tab_cta_bad}
  \begin{tabular}{|l|c|cc|c|ccc|}
    \hline
    &
    \shortstack{大口径望遠鏡 \\ Large-Sized Telescope \\ (LST)} &
    % 複数列を結合したいときは、multicolumnを使います。
    \multicolumn{2}{c|}{\shortstack{中口径望遠鏡 \\ Medium-Sized Telescope \\ (MST)}} &
    \shortstack{SC 中口径望遠鏡 \\ Schwarzschild--Couder MST \\ (SC-MST)} &
    \multicolumn{3}{c|}{\shortstack{小口径望遠鏡 \\ Smalle-Sized Telescope \\ (SST)}} \\
    \hline
    & & FlashCam & NectarCAM & & GCT & ASTRI & 1M-SST \\
    \hline
    エネルギー範囲 & 20--200 GeV & \multicolumn{2}{c|}{100 GeV -- 10 TeV} & 200 GeV -- 10 TeV & \multicolumn{3}{c|}{5--300 TeV} \\
    \hline
    台数(北半球)& 4 & \multicolumn{2}{c|}{15} & 0 & \multicolumn{3}{c|}{0} \\
    \hline
    台数(南半球)& 4 & \multicolumn{2}{c|}{24} & 24 & \multicolumn{3}{c|}{70--90} \\
    \hline
    鏡直径 &	23~m & \multicolumn{2}{c|}{12~m} & 9.7~m & 4~m & 4~m & 4~m \\
    \hline
    焦点距離 & 28~m & \multicolumn{2}{c|}{16~m} & 5.6~m & 2.3~m & 2.15~m & 5.6~m \\
    \hline
    視野 & 4.5$^\circ$ & \multicolumn{2}{c|}{7.7$^\circ$} & 8$^\circ$ & 8.6$^\circ$ & 9.6$^\circ$ & 9$^\circ$ \\
    \hline
    光学系 & 放物鏡 & \multicolumn{2}{c|}{Davies--Cotton (DC)} & Schwarzschild--Couder (SC) & SC & SC & DC \\
    \hline
    画素数 & 1,855 & 1,764 & 1,855 & 11,328 & 2,048 & 1,984 & 1,296\\
    \hline
  \end{tabular}
  \normalsize % 文字サイズを元に戻します
\end{table}

\section{数式の使い方}

\LaTeX{}を使う理由のひとつが、数式を綺麗に出力できるというのがあります。例えば中性パイ中間子$\pi^0$のガンマ線への二体崩壊であれば
\begin{equation}
  \pi^0 \rightarrow \gamma + \gamma
\end{equation}
のように書けますし、もっとややこしい数式も色々と書けますが、詳細は「LaTeX 数式」などでインターネット上で検索してください。この例のように、本文中に数式を入れるときは\texttt{\$\$}でその式を囲み、独立した行に数式を書くときは\texttt{equation}や\texttt{align}\footnote{\texttt{amsmath}パッケージで使用可能です。}環境を使ってください。

\subsection{斜体と立体}
数式を書くときには「斜体」(italic)と「立体」(upright)の違いに気をつけてください。基本的に数式は斜体を使って書きます。何も考えずに\LaTeX{}を使えば全て斜体になります。

ただし、次の2つの式を見比べてみてください。
\begin{equation}
  e^{ix}=cosx + isinx
  \label{eq_italic}
\end{equation}
\begin{equation}
  \mathrm{e}^{\mathrm{i}x}=\cos x + \mathrm{i}\sin x
  \label{eq_upright}
\end{equation}
式~\ref{eq_italic}は全ての文字が斜体で書かれていますが、式~\ref{eq_upright}は$x$以外は立体です。このように、いくつかの文字では立体を使うのが一般的です。例えば$\log$、$\sin$、$\mathrm{e}$(自然対数の底)、$\mathrm{i}$(虚数単位)、$\mathrm{d}$(微分作用素)などは、それぞれ\texttt{\bs{}log}、\texttt{\bs{}sin}、\texttt{\bs{}mathrm\{e\}}、\texttt{\bs{}mathrm\{i\}}、\texttt{\bs{}mathrm\{d\}}などと入力することで書くことができます\footnote{自然対数の底や虚数単位の場合は、分野や国によって斜体にするかどうかの違いがあります。また微分作用素は斜体で$d$とする場合もありますが、立体にすることで長さを表すのに頻繁に使われる変数$d$と区別する効果があります。}。

ここで\texttt{\bs{}mathrm}というコマンドが出てきましたが、これは数式中で文字を立体にするためのコマンドです。特定の文字を立体にするときだけでなく、変数名の添字を立体するときにも使います。例えばトリガー回数を示す変数は$N_\mathrm{trigger}$などと書くことがあると思いますが、このときに「trigger」の部分は変数ではありませんので、斜体にしません。

\subsection{単位}
数式中に単位を使うとき、\texttt{\bs{}mathrm}を使わずに$100 MeV$などとしてしまう間違いもよく見られます。このように斜体になったものは変数$M$と$e$と$V$の掛け算であり、単位ではありません。また100MeVのように単位と数値の間にスペースのない書き方をする人も見かけますが、これも間違いです。本文中に書くときは\texttt{100\~{}MeV}とし、\texttt{equation}環境中では\texttt{100\bs{} \bs{}mathrm\{MeV\}}と書きます\footnote{余計なバックスラッシュとスペースは、数字と単位の間にスペースを入れるためです。}。

\LaTeX{}では\texttt{\%}の後ろをコメントとして扱いますので、95\%のようにパーセントの表示をしたい場合には\texttt{95\bs{}\%}のように書きます。\%と数値の間にスペースを入れるかどうかは、流派が2つありますが、私の周りでは入れない人が多いようです\footnote{入れない理由としては、\%は単位ではなく0.01という数だから、というものが挙げられます。}。

\section{引用の仕方}

研究や論文というのは過去に誰かのやった研究を前提として新たに何かを進歩させるためにあります\footnote{「巨人の肩の上に立つ」とよく表現されます。}。しかしあなたの修士論文に全ての過去の研究を書くことはできませんので、引用という形式を使い他の論文をその事実の出典とします。

ここで、「引用」と日本語で書いた場合には「quotation」と「citation」の2つの英語に翻訳され得ますが、我々の論文で通常用いるのは「citation」のほうです。著作権法などで問題になるのは「quotation」のほうなので、間違えないようにしてください。

\LaTeX{}で\texttt{citep}コマンドや\texttt{citet}コマンドを使って論文を引用(cite)するときは、例えば次のようになります。

\begin{quote} % ここで LaTeX の体裁を整えるために quote コマンドを使っていますが、ここは「引用」(quote)しているわけではりません。
  宇宙線の全粒子スペクトルは図 XX に示すように$10^9$~eVから$10^{20}$~eVまでおよそ$-3$乗の冪で減少している\citep{Swordy2001}。$10^{12}$~eV(1~TeV)付近のガンマ線は超高エネルギーガンマ線と呼ばれ、様々な観測手法が提案されている\citep[例えば][を見よ]{Okumura2005}。この\citet{Okumura2005}の手法では\ldots
% 複数の文献を一度に引用するには \citep{Swordy2001,Okumura2005}とすることもできます。
\end{quote}
ここでは引用(cite)を3回しており、それぞれ\texttt{citep}、\texttt{citep}、\texttt{citet}コマンドを使っています。

\section{\BibTeX{}の使用}
このテンプレートの場合、\pageref{page:bib}ページに「引用文献」という箇所があります。このページを手作業で間違いなく整形するのは面倒です。手でやる代わりに\BibTeX{}という仕組みを使います。\texttt{thesis.bib}というファイルに引用文献の必要な情報が書かれていますので、これを参考にして\BibTeX{}ファイルを作るか、論文をダウンロードするときに\texttt{.bib}ファイルもダウンロードできますので、それを使ってください\footnote{近年は「文献管理ソフト」と呼ばれるものが発達していますので、特に博士進学する学生は好きなものを入れてみてください。}。

\section{ヨーロッパ圏の人名など}
ウムラウトなどの混じったヨーロッパ圏の人名を入力するには、例えばシュレーディンガーの場合、\LaTeX{}では\texttt{Shr\bs{}"\{o\}dinger}と入力することでShr\"{o}dingerと表示すると\LaTeX{}の教科書には書いてあります。しかしいちいちこんなことをするのは面倒ですので、\texttt{main.tex}に書いてある\texttt{\bs{}usepackage[utf8]\{inputenc\}}を使うことで、直接ウムラウトつきの文字を\LaTeX{}のソース中に書いてしまって問題ありません。「ö」と「\"{o}」は、この\LaTeX{}ソース中では違う入力方法で書かれていますが、出力は同一です。

\section{\texttt{newcommand}}

入力が長く、論文中で何度も繰り返し使うような入力はコマンドとして登録することができます。例えば\texttt{\bs{}HI\{\}}や\texttt{\bs{}bs\{\}}といったコマンドを\texttt{main.tex}で定義しており、これらの結果は「\HI{}」や「\bs{}」と表示されます。

\chapter{剽窃について}

\section{剽窃とは何か}
\label{sec:plagiarism}
「剽窃(ひょうせつ)」とは
\begin{itemize}
\item 「他人の作品や論文を盗んで、自分のものとして発表すること。」『大辞泉』
\item 「他人の作品・学説などを自分のものとして発表すること。」『スーパー大辞林』
\item 「他人の著作から,部分的に文章,語句,筋,思想などを盗み,自作の中に自分のものとして用いること。他人の作品をそっくりそのまま自分のものと偽る盗用とは異なる。」『ブリタニカ国際大百科事典 小項目事典』
\end{itemize}
のように辞書では説明されています。

例えばここで『ブリタニカ国際大百科事典 小項目事典』を引用元として明記せずに、
\begin{quotation}
  \red{剽窃(ひょうせつ)とは、}他人の著作から\red{、}部分的に文章\red{、}語句\red{、}筋\red{、}思想などを盗み\red{、}自作の中に自分のものとして用いること\red{です}。他人の作品をそっくりそのまま自分のものと偽る盗用とは異な\red{ります}。
\end{quotation}
という説明をしたとします。これが剽窃です。この例では赤字で示したとおり、文体をですます調に変更したり、読点を「,」から「、」に変更したり、文頭に「剽窃(ひょうせつ)とは、」と書き加えたりしていますが、全体としては同一の文章であるため、通常は剽窃と見なされます。

学術論文ではない創作物の形態によっては、剽窃行為が「インスパイア」や「オマージュ」という言葉で括られることもあります。しかし修士論文での剽窃行為は不正行為です。試験でのカンニングやレポートの丸写しと同じであり、(まともな大学や研究室であれば)厳しく罰せられます。

\section{剽窃をするとどうなるか}

修士論文中に剽窃行為が発見された場合、その学期における単位をすべて没収され、卒業に必要な単位が与えられず修士課程を修了できなくなる可能性が高いです。各大学や研究科でどのような対応を実際に取るかはそれぞれだと思いますが、少なくとも私が審査員を担当した場合には落第させます。

修論審査に落第すれば、もし就職が決まっていても留年を余儀なくされます。留年を選択せず修了を諦めて中退するにしても、就職先は剽窃行為のせいで修了できなかった学生をそのまま採用はしてくれないでしょう。仮に同じ企業に就職が認められたとしても、修士卒扱いで入社できたはずのところが学部卒扱いとなり、初任給が月額数万円低い状態から開始となります。例えば同期と2万円の月給差を保ったまま40 年間働くとすると生涯収入で 1000 万円程度の損失になります。もし留年する道を選んでも、定年時点で1000万円程度の年収を見込めるのであれば、生涯収入としてその額だけ失うことになります。

もし博士課程に進学する場合、なぜ留年したかの説明を陰に陽に常に求められます。たとえ直接にその理由を問われることがなくとも、他の学生より1年多く修士課程に時間がかかったということは、優秀な学生ではないと周りから見なされ、研究をする上でも奨学金などを取得する上でも不利になるでしょう。また標準年限を超えての在籍の場合、大学院の授業料免除などの制度も利用できなくなる可能性があります。

\section{修士論文における剽窃について}
節\ref{sec:plagiarism}に引用した一般的な剽窃の定義ではなく、科学文書や、特に修士論文での剽窃についてもう少し踏み込んで説明し直してみましょう。

\subsection{いわゆるコピペ}

少なくとも宇宙物理学分野における修士論文は独自性のあるものでなくてはいけません。独自性のある(オリジナル)とは次のようなことです。
\begin{itemize}
\item 誰かが過去にやった研究ではないこと
\item 自分自身の手でやった研究であること(共同研究であれば、十分に自分の貢献のあること)
\item 研究本体以外の章も含め、すべて自分の言葉で説明できること
\end{itemize}

したがって、誰かの論文や教科書の記述をそっくりそのまま持ってきて(いわゆる「コピペ」して)、それを自分の修士論文として提出することは許されません。高校や大学のレポートなどでも、他人のレポートを写すなと散々注意されるのと同じことです。

これはコピペする文章の長さに依りません。たとえ1行であってもコピペはコピペであり、剽窃と見なされます\footnote{ただし、ごくありふれた表現や、酷似するのが避けられない科学的事実は除く。}。

もちろん、ある文章を他の論文や書籍から引用(quote)する必要のある場合は、逆に改変してはいけません。そっくりそのまま書き写し、それを自分の文章とは別のものであると分かるように引用符や枠で囲むなりします。しかし宇宙物理学関連の修士論文でこのような引用をすることは、ほとんどないと思います。

\subsection{他人の文章の改変}

コピペとともによく見られるのが、他人の文章を一部だけ改変して自分が書いたかのように装うことです。完全に同一のものを持ってくる方が簡単ですし、なぜこのような行動を取るのかよく分かりませんが、私の経験として最も多い剽窃行為がこの文章の一部改変です。

もしかすると「先輩の修論を写したりコピペするなよ。自分の言葉で書けよ」とだけ教員から指導を受けると、表面的に一部改変すれば剽窃にはならないと勘違いするのかもしれません。しかし元の文章が存在しなければ作成できないのですから、これは独自性のある文章とは見なされず、やはり剽窃行為となります。

たとえば次のような文章が「元ネタ」として存在していたとしましょう\footnote{これはきちんと添削を受けていない、今となっては恥ずかしい私の修論の一節ですが、あくまで例です。}。

\begin{quotation}
  1910年代にHessらによって宇宙線の存在が確認されて以来、様々なエネルギー領域、様々な検出器によって宇宙線の観測が行われてきた。同時に、ガリレオ以来発達してきた可視光による天体の観測も、電波望遠鏡や赤外望遠鏡の登場によって多波長での観測へと発展することとなった。

  宇宙線と言っても、その成分は電磁波、陽子、原子核、neutrinoなど様々であり、それらの持つエネルギーも広範にわたる。現在地球上で確認されている宇宙線のうち、最もエネルギーの高いものは$10^{20}$~eVを超える(最高エネルギー宇宙線)。これは人工的に到達できるエネルギーを実に8桁も上回るが、なぜそのような高エネルギーの宇宙線が存在するのかは謎である。加速機構、地球までの伝播過程、1次宇宙線成分は何であるのか、いずれも未解明のままであり、その興味は尽きない。
  \flushright{\citet{Okumura2005}より引用}
\end{quotation}

少しこれを改変してみましょう。赤字が削除箇所、青字が追加箇所です。実際に私が発見してきた剽窃行為には、このような改変が多くありました。
  
\begin{quotation}
\DIFdelbegin \DIFdel{1910年}\DIFdelend \DIFaddbegin \DIFadd{1912}\DIFaddend \DIFdelbegin \DIFdel{代}\DIFdelend \DIFaddbegin \DIFadd{年}\DIFaddend に\DIFaddbegin \DIFaddend Hess\DIFdelbegin \DIFdel{ら}\DIFdelend によって宇宙線\DIFdelbegin \DIFdel{の存在}\DIFdelend が\DIFdelbegin \DIFdel{確認}\DIFdelend \DIFaddbegin \DIFadd{初めて発見}\DIFaddend されて以来、\DIFdelbegin \DIFdel{様々な}\DIFdelend \DIFaddbegin \DIFadd{広い}\DIFaddend エネルギー\DIFdelbegin \DIFdel{領域}\DIFdelend \DIFaddbegin \DIFadd{範囲}\DIFaddend 、\DIFdelbegin \DIFdel{様々}\DIFdelend \DIFaddbegin \DIFadd{多種多様}\DIFaddend な検出器によって宇宙線\DIFdelbegin \DIFdel{の}\DIFdelend \DIFaddbegin \DIFaddend 観測が行われてきた。\DIFdelbegin \DIFdel{同時に}\DIFdelend \DIFaddbegin \DIFadd{また}\DIFaddend 、ガリレオ以来発達してきた可視光\DIFdelbegin \DIFdel{による天体の観測}\DIFdelend \DIFaddbegin \DIFadd{での天体観測}\DIFaddend も、電波望遠鏡や赤外望遠鏡\DIFaddbegin \DIFadd{という新しい観測手段}\DIFaddend の登場\DIFdelbegin \DIFdel{によって多波長での観測}\DIFdelend \DIFaddbegin \DIFadd{により、多波長観測}\DIFaddend へと発展\DIFdelbegin \DIFdel{することとなった}\DIFdelend \DIFaddbegin \DIFadd{した}\DIFaddend 。

宇宙線と\DIFdelbegin \DIFdel{言}\DIFdelend \DIFaddbegin \DIFadd{い}\DIFaddend っても、その成分は\DIFdelbegin \DIFdel{電磁波、}\DIFdelend 陽子、原子核、\DIFdelbegin \DIFdel{neutrino}\DIFdelend \DIFaddbegin \DIFadd{電子、ニュートリノ}\DIFaddend など様々であり、\DIFdelbegin \DIFdel{それらの持つ}\DIFdelend \DIFaddbegin \DIFadd{その}\DIFaddend エネルギー\DIFaddbegin \DIFadd{範囲}\DIFaddend も\DIFdelbegin \DIFdel{広範}\DIFdelend \DIFaddbegin \DIFadd{何桁}\DIFaddend に\DIFaddbegin \DIFadd{も}\DIFaddend わたる。現在\DIFdelbegin \DIFdelend \DIFaddbegin \DIFadd{、}\DIFaddend 地\DIFdelbegin \DIFdel{球}\DIFdelend 上で確認されている宇宙線のうち、最もエネルギーの高いものは$10^{20}$~eVを超える(\DIFaddbegin \DIFadd{いわゆる}\DIFaddend 最高エネルギー宇宙線)。これは\DIFdelbegin \DIFdel{人工的に}\DIFdelend \DIFaddbegin \DIFadd{加速器で人類が}\DIFaddend 到達できるエネルギーを\DIFdelbegin \DIFdel{実に}\DIFdelend 8桁も上回るが、なぜそのような高\DIFaddbegin \DIFadd{い}\DIFaddend エネルギーの宇宙線が存在するのかは\DIFdelbegin \DIFdel{謎である}\DIFdelend \DIFaddbegin \DIFadd{解明されていない}\DIFaddend 。\DIFaddbegin \DIFadd{宇宙線の}\DIFaddend 加速機構、地球までの伝播過程、\DIFaddbegin \DIFadd{また}\DIFaddend 1次宇宙線成分は何であるのか\DIFaddbegin \DIFadd{は}\DIFaddend 、いずれも未解\DIFdelbegin \DIFdel{明}\DIFdelend \DIFaddbegin \DIFadd{決}\DIFaddend の\DIFdelbegin \DIFdel{まま}\DIFdelend \DIFaddbegin \DIFadd{問題}\DIFaddend であり、\DIFdelbegin \DIFdel{その興味は尽きない}\DIFdelend \DIFaddbegin \DIFadd{将来の宇宙線観測計画による解決が期待される}\DIFaddend 。
  \flushright{\citet{Okumura2005}を意図的に改変}
\end{quotation}

\subsection{元の文章を下敷きに自分で考えたつもりになったもの}

さらに改変の量を増やし、ところどころに自分の独自の文を入れたり、文の前後を入れ替える剽窃もあります。自分で考えて文を挿入するのだから剽窃ではないと考える人もいるかもしれませんが、やはり元の文章が存在しなければ書くことのできない文章ですので、これも立派な剽窃です。たとえば次のようなものです。

\begin{quotation}
  Hessの気球実験によって1912年に宇宙線が大気中で発見されてから、様々な粒子、多様な検出手法、またMeV領域から$10^{20}$~eVにまでおよぶエネルギー範囲で宇宙線の観測が行われてきた。一方、電磁波による天体の観測も、ガリレオによる可視光観測に始まり、電波望遠鏡や赤外線望遠鏡などの登場によって他波長観測へと発展した。さらに近年の重力波やニュートリノ観測を加え、現在の宇宙観測は、多粒子、他波長観測の時代、すなわちマルチメッセンジャー天文学へと進展した。

  このうち宇宙線は、陽子、原子核、電子、ニュートリノなどを含む、宇宙空間を飛び交う高エネルギーの粒子である。先に述べたように、その最高エネルギーは$10^{20}$~eVにまでわたる(いわゆる最高エネルギー宇宙線)。これは人類がLHC加速器で到達できる数~TeVというエネルギーを8桁も上回るものであるが、なぜそのような高いエネルギーの宇宙線が宇宙で加速されているのか、宇宙線の発見から100年以上が経っても未解決の問題である。その加速機構、加速天体、地球までの伝播、また粒子の種類がなんであるかという謎を解き明かすには、今後の宇宙線観測手法に大きな飛躍が必要である。
  \flushright{\citet{Okumura2005}を意図的に改変}
\end{quotation}

ここまで改変すると、全く違う文章のように感じる人もいるかもしれませんが、実際に行われる剽窃行為では、このような元ネタに改変を加えた文章が何段落も続くことが多いです。そのため、文章の一部が似通っているだけでなく、その章の論理展開自体がほとんど同じになってしまうのです。

研究背景は過去に行われた研究の積み重ねなので、論理展開が同じになることは仕方がないという主張をする学生もいます。しかし修士論文はその研究目的が各々違うわけですから、論文のイントロなどで全く同じ論理展開になることは本来ありえません。その論文独自の研究内容を説明するためにイントロは書かれるべきであり、他の文章と同じであるというのは、イントロを書くという目的を勘違いしています。

\subsection{出典のない図表の使用}

他人の文章を剽窃する行為とは別に、図表を適切に引用(cite)せずに流用するという剽窃もあります。これは悪意があって行われているわけではなく、引用の作法を知らないだけの場合が多いため罪としては軽いかもしれません。しかし、その修士論文の読者に対して「この図は自分が作りました」と嘘をつくのと同じ行為ですので、やはり問題行為であることは理解できると思います。

このような図表の剽窃は、特に共同研究で多く見られます。ある実験プロジェクトに参加している場合、実験装置の説明の図や写真をプロジェクト内で使いまわすことがあるでしょう。たとえば図\ref{fig_CTA}のようなものが該当します。もしこれを出典もしくは作者を明記せずに使用した場合、剽窃行為に当たります\footnote{おそらく「出典を明記して再提出しろ」と言われるだけで、落第はしないと思いますが。}。

図表の提供者の名前を入れる、その図が最初に使われた論文や出版物が存在する場合はそれを出典として明記する(cite する)、もしくは提供した実験グループなどの名前を入れるなどしてください。

\subsection{アイデアの盗用}
他人の考えた研究アイデアを自分が考えたかのように記述するのも剽窃です。例えば投稿論文になっていないものの、先輩の修士論文で先行研究が行われていたとしましょう。これを先行研究として取り上げることなく、「〜〜という手法を本論文では考案し」などと書くのは剽窃行為です。きちんと「〜〜という手法が先行研究で提案され、本論文ではこれを発展させ」のように書きましょう。

\subsection{自己剽窃}

自己剽窃とは、自分の書いた論文などから図や文章を剽窃して再利用することです。なぜこれが問題とされるのか、直感的にはすぐに分からないかもしれません。

自己剽窃が最も問題とされるは、論文の二重投稿です。どこかで論文を出版する場合、レビュー論文でない限り、それぞれが独自の新規性を持つ論文でなくてはいけません。したがって、業績稼ぎのために同じ内容の論文を複数の場所で発表するのは研究不正として扱われます。

次に自己剽窃が問題となるのは、著作権の問題です。投稿論文を科学誌に掲載する多くの場合、その著作権を出版社に譲渡することになります。最近のオープンアクセス(open access)誌の場合には著作権が論文著者に残される場合もありますが、投稿論文の著作権を必ずしも自分が持っているわけではないのだということを覚えておいてください。

著作権が出版社にあるということは、その著作物を引用の範囲を超えて勝手に再利用してはいけないということになります。著作権、英語で書くと copyright ですが、すなわち複製する権利を出版社に譲渡してしまっているからです。

ただし、多くの出版社では学位論文や国際会議のプロシーディングスなどで、著者が図表などを出版社に断らずに使いまわすことを許可しています。ただし、出典を明記することは求められていることが多いはずです。もし投稿論文に使用した図表もしくは文章を修士論文で使いまわす場合、出版社との著作権の契約について理解しておきましょう。たとえば Elsevier 社の場合、\url{http://jp.elsevier.com/authors/author-rights-and-responsibilities} に著者の権利が書かれています。他の出版社も同様の情報を公開しています。

\section{なぜ剽窃は許されないのか}

なぜ剽窃行為は許されず、それが修士論文で不正行為とされるのか、その理由を改めてまとめます。

\begin{enumerate}
\item 学位審査は、学生が研究背景などを理解しているか、またそれを自分の言葉で伝える能力を身につけているかを審査する場です。したがって、剽窃を含む文書ではこの審査を適切に行えなくなってしまいます。修士の学位を与える審査の一環として修士論文を執筆しているわけですから、修士論文作成能力がないのにそれを他人の文章を使って誤魔化すのは、当然不正行為になります。

\item 同じ文章を使いまわすとき、一般的には引用 (cite ではなくて quote) をし、自分の書いた文章と他人の文章を区別するのが標準的です。超新星の過去の記録など一部の例を除き、宇宙物理学分野でquoteのほうの引用をすることはほとんどありません。もし必要となる場合は、他人の書いた文章であることが明確に読者に分かるようにしましょう。自分で作った文章かのように見せるのは決して許される行為ではありません。

\item 他人の書いた文章を自分が書いたかのように見せるのは、人の手柄を横取りすることになります。

\item 少なくとも日本の国内においては、他人の著作物を勝手に使用したり改変したりすることは、著作権の侵害に当たる行為です。

\item 元の文章を無理に改変することにより、推敲された元の文章よりも質の低い文章になることが多く、また間違った記載となる場合が多々あります。例えば「突発天体を観測する」を無理やり「突発天体を監視する」に変更することにより、意味が大きく変わることもあります。

\item 同じものを繰り返すというのは、先人の研究をさらに発展させていくという、科学の営み自体を否定する行為です。

\item 過去数年で該当分野に大きな進展があった場合にも、それを無視した様な文章が生産されてしまいます。例えば 2018 年の修論なのに重力波が未だ検出されていない前提の文章になっていたりということが考えられます。

\item 修論の添削をする教員は、執筆した学生の研究能力や文章作成能力を高めるために添削をしています。良い出来の修論を書かせることが目的ではないのです。そのため、本人が書いてすらいない文章を添削させ、大学教員の貴重な時間を奪うことは、学生と教員の間の信頼関係を大きく毀損する大変失礼な行為です。またそのような添削をしても本人が書いていないのですから、その学生の能力向上には全く役に立たず、学生も自分で考えることなく言われるがままに改訂を繰り返すことになるでしょう。

\end{enumerate}

\chapter{色覚多様性と作図}
\label{chap:color}

\section{色覚多様性とは}

「色覚多様性」や「色覚特性」という言葉を聞いたことがあるでしょうか。もし聞いたことがなくても、「色覚異常」「色盲」「色弱」という言葉であれば知っているかもしれません\footnote{これら言葉の使用の是非についてはこの文章の範囲を超えるので論じませんが、本章では「色覚多様性」という表現で統一します。左利きの人や AB 型の人ををわざわざ「利き手異常」「血液型異常」と呼ばないように、「異常」のような言葉の使用を避ける人が多いということは知っておいてください。}。人間の色覚は人によって異なり、あなたの目で見ている色の見え方が他の人とは異なる場合があります。このような多様性を、色覚多様性と言います。

人によって異なると言っても、人間の色覚はいくつかの種類に分類されます。血液型が人によって違ったり、利き手の左右が違ったりと同様です。例えば日本人男性の場合、その約5\%は赤と緑の色の差を区別しにくいという色覚を持っていると言われています。日本人女性であれば0.2\%、白人男性であれば8\%程度とも言われています。

図\ref{fig_color_line1}は 2 つの関数を色の違いのみで描いたものです。両方とも実線を使っていますが、$\sin(x)$ は赤、$\cos(x)$ は緑を使用しています。あなたが運よく多数派の(「正常」と呼ばれる)色覚特性を持つのであれば、これら 2 つの関数の区別は問題なく行えるでしょう。しかし赤と緑の区別が困難な人には、同じものが図\ref{fig_color_line2}のように見えている可能性があります。

\begin{figure}
  \centering
  \subfigure[]{%
    \includegraphics[width=.5\textwidth,clip]{fig/color_line1.pdf}%
    \label{fig_color_line1}%
  }%
  \subfigure[]{%
    \includegraphics[width=.5\textwidth,clip]{fig/color_line2.pdf}%
    \label{fig_color_line2}%
  }
  \caption[色覚特性による図の見え方の違い]{色覚特性による図の見え方の違い。(a) 「正常」とされる場合の人は、この赤と緑の区別がつく。(b) 赤と緑の区別がつきにくい色覚特性を持つ人には、このように見える(シミュレーション)。}
\end{figure}

\section{色覚多様性を配慮した作図}

\subsection{線種の変更}
それでは、図\ref{fig_color_line1}に示したような2つの関数を誰にでも区別できるようにするには、どのようにすれば良いでしょうか。曲線や直線を使用している図の場合、線種を変更するのが一般的に行われる手法です。図\ref{fig_color_line3}では色の違いに加えて線種の違いを加えました。これであれば色の違いが分からなくても、図\ref{fig_color_line4}のように2つの関数を区別できるようになります。一度に表示する線の数が多い場合、実線、破線、点線、一点鎖線、太線、細線などと使い分けましょう。

\begin{figure}
  \centering
  \subfigure[]{%
    \includegraphics[width=.5\textwidth,clip]{fig/color_line3.pdf}%
    \label{fig_color_line3}%
  }%
  \subfigure[]{%
    \includegraphics[width=.5\textwidth,clip]{fig/color_line4.pdf}%
    \label{fig_color_line4}%
  }
  \caption[色覚特性を考慮した作図例]{色覚特性を考慮した作図例。(a) 色の違いに加えて、線種も変更した。(b) 赤と緑の区別ができなくても、線種によって区別が可能。}
\end{figure}

\subsection{マーカー形状の変更}

散布図のようにデータ点を表示する場合、印(マーカー)の形状を変更します。赤も緑も青もよく使う色ですので図\ref{fig_color_marker1}のような図は修論でも学会でもよく見かけます。しかし、例えば赤と緑の区別がつかない色覚特性の場合、これは図\ref{fig_color_marker2}のように化けます。最悪ですね、なにも区別できなくなります。

これはマーカーの形状を変更することで、図\ref{fig_color_marker4}のように容易に区別が着くようになります。測定値を折れ線グラフで結ぶような場合は、さらに折れ線の線種も変更することでより区別しやすくなる場合があります。

\begin{figure}
  \centering
  \subfigure[]{%
    \includegraphics[width=.5\textwidth,clip]{fig/color_marker1.pdf}%
    \label{fig_color_marker1}%
  }%
  \subfigure[]{%
    \includegraphics[width=.5\textwidth,clip]{fig/color_marker2.pdf}%
    \label{fig_color_marker2}%
  }
  \subfigure[]{%
    \includegraphics[width=.5\textwidth,clip]{fig/color_marker3.pdf}%
    \label{fig_color_marker3}%
  }%
  \subfigure[]{%
    \includegraphics[width=.5\textwidth,clip]{fig/color_marker4.pdf}%
    \label{fig_color_marker4}%
  }
  \caption[散布図の作図例]{散布図の作図例。(a) 色の違いしかない場合。(b) 色の違いしかない場合の見え方の例(シミュレーション)。(c) 色の違いに加えて、マーカー形状も変更した。(d) 赤と緑の区別ができなくても、マーカー形状によって区別が可能。}
\end{figure}

\subsection{グラデーションの変更}

2次元のヒストグラムのように、線とマーカーだけでは値の増減を可視化できないものも存在します。そのようなものは色のグラデーションを使用しますが、これも注意深く選定しないと、人によっては同じ色がグラデーション中に複数回登場するように見えてしまう場合があります。

図\ref{fig_color_palette1}は虹色のグラデーションで$\sin(x)\cos(x)$を表示したものです。このグラデーションでは赤系の色と緑系の色が黄色($0.25$のあたり)を挟んで並んでいます。したがって、$0.25$ 周辺で値が変化する場合には、それが増えているのか減っているのか識別困難になります。

図\ref{fig_color_palette2}を見てみましょう。この図で左上と左下の領域が、増えているのか減っているのか区別がつくでしょうか。赤系と緑系の区別ができる人には、図\ref{fig_color_palette1}で黄緑からオレンジへの色の遷移が読み取れます。しかし、これらの色が区別できない場合、この増減は図\ref{fig_color_palette2}から読み取ることはできないのです。

\begin{figure}
  \centering
  \subfigure[]{%
    \includegraphics[width=.5\textwidth,clip]{fig/color_palette1.png}%
    \label{fig_color_palette1}%
  }%
  \subfigure[]{%
    \includegraphics[width=.5\textwidth,clip]{fig/color_palette2.png}%
    \label{fig_color_palette2}%
  }
  \caption[グラデーションの見え方の例]{グラデーションの見え方の例。(a) 同程度の輝度で赤系と緑系の色が出てくるグラデーションの使用例。(b) 赤系と緑系の区別がつかない場合の見え方の例(シミュレーション)。例えば左上や左下の領域が増加しているのか減少しているのか、かなり注意深く見ても識別するのが難しい。}
\end{figure}

\subsection{色覚多様性を配慮した図になっているかの判断}

あなたが「正常」な色覚特性を持っている場合、他の人がどのように見えているかを即座に想像するのは困難です。したがって、自分の作図が誰にとっても読みやすいかを知るのは容易ではありません。しかし大原則として、「その図を白黒で印刷しても自分が正しく読み取れるか」という考え方をするのが、最も安全な判断方法だと思います。

また、色覚多様性をシミュレーションするソフトウェアも近年は無料で利用することが可能です。例えば Mac や iOS であれば「Sim Daltonism」というアプリケーションが入手可能です。図\ref{fig_color_palette2}は Mac 版の Sim Daltonism を使って見え方をシミュレーションしたものです。Windows や Android でも同様のアプリケーションがあるので、探してみてください。

\begin{itemize}
\item macOS 用の Sim Daltonism \url{https://apps.apple.com/jp/app/sim-daltonism/id693112260}
\item iOS 用の Sim Daltonism \url{https://apps.apple.com/jp/app/sim-daltonism/id1050503579}
\end{itemize}

\chapter{議論}
ここではこの研究で得られた結果についての議論を行います。測定結果や観測結果などと一緒に議論を進める場合もあるので、必ず必要な章であるとは言えませんが、できる限り研究で得られた事実と自分の議論は分けましょう。

\chapter{結論}
ここには自分の修士論文の結論を書きます。「議論」の章で書かれたことも、再びここに短く書かれます。

「序論」で始めたら「結論」、「はじめに」で始めたら「おわりに」が原則です。ただし、「まとめと今後の展望」などとすることもありますので、好みに応じて変えてください。

\chapter*{付録} % 章番号を出さない
\addcontentsline{toc}{chapter}{付録} % 目次に載せる

「付録」(appendix)は、論文の本文に載せるには情報として邪魔もしくは必須ではないものの、読者にとって有益となるような情報を載せます。付録を必要としない論文ももちろん存在しますので、そこは著者の判断です。

例えば、たくさんの観測データを様々なモデルでフィットした場合、フィット結果の絵がたくさん出てくるはずです。そのような図は本文中に大量に出されても大切な情報を見失ってしまいますので、大部分は付録に載せることが推奨されます。他には、何かしらの長い式変形や証明を載せる必要がある場合、付録に移動する場合があります。

% 付録は chapter の 1 つとして作りますが、章番号は表示しません。
% また付録の 1 つずつはアルファベットで番号付けをするのが一般的です。
\setcounter{section}{0} % section の番号をゼロにリセットする
\renewcommand{\thesection}{\Alph{section}} % 数字ではなくアルファベットで数える
\setcounter{equation}{0} % 式番号を A.1 のようにする
\renewcommand{\theequation}{\Alph{section}.\arabic{equation}}
\setcounter{figure}{0} % 図番号
\renewcommand{\thefigure}{\Alph{section}.\arabic{figure}}
\setcounter{table}{0} % 表番号
\renewcommand{\thetable}{\Alph{section}.\arabic{table}}

\section{すごい長い証明}
式~(\ref{eq})のように、式番号がアルファベットとアラビア数字の組み合わせになるように、\LaTeX{}ソース中で設定してありますので、中身を眺めてみてください。

\begin{equation}
  \label{eq}
  1 + 1 = 2
\end{equation}


\section{すごいたくさんのフィットの図}

\section{修士論文添削前に自己点検する項目}

\begin{itemize}
\item[\CID{00728}] 第\ref{chap:plagiarism}章を読み、剽窃について十分に理解したか。
\item[\CID{00728}] 修士論文に剽窃箇所もしくは剽窃と見なされうる箇所は存在しないか。
\item[\CID{00728}] \LaTeX\ で図番号などの参照先がないせいで「図??」「表??」「??節」のようになっている箇所はないか。
\item[\CID{00728}] 日本語読点「、」と欧文カンマ「,」が混在していないか。例えば「ガンマ線望遠鏡は、HESS, MAGIC, VERITASなどがある」。
\item[\CID{00728}] 日本語丸括弧「()」と欧文丸括弧「()」が混在していないか。
\item[\CID{00728}] 単位と数値の間にスペースは入っているか。「100MeV」など。
\item[\CID{00728}] 単位が斜体になっていないか。「$100~MeV$」など。
\item[\CID{00728}] 変数でない添字などが斜体になっていないか。$N_{trigger}$など。
\item[\CID{00728}] 自分で作成したものではない図や写真は、全て出典が明記され、転載であることを書いてあるか。
\end{itemize}

\chapter*{謝辞}%
\addcontentsline{toc}{chapter}{謝辞}

一般的に論文における「謝辞」(acknowledgments)とは、その論文を作成する上で不可欠だった様々な支援に対する感謝の気持ちを述べる場所です。通常の投稿論文であれば、その論文の作成者・共同研究者は主著者(筆頭著者もしくは責任著者)や共著者として著者リストに入っています。しかし修士論文は単著で書くため、例えば指導教員から論文のアイデアをもらっても指導教員は共著者になりません。また様々な共同研究者にデータ解析を手伝ってもらったり、実験データを提供してもらった場合にも、これが普通の論文であれば共著者になりうるところですが、修士論文では単著、つまりあなたの名前だけが記載されます。そこで、謝辞の必要性が生じるのです\footnote{複数著者による投稿論文の場合、共著者にするべきか謝辞での言及のみに留めるかは、場合によります。}。

先輩の修論の謝辞を真似て、やたらと大人数への謝辞を並べてある修論をよく見かけます。私の所属する研究室は大所帯のため、教員全員、院生全員、事務補佐員が20人以上並んでいるものがあります。もちろん感謝を述べたかったら書くのは構いませんしそれを止めるつもりはありませんが、修論の読者からすると「別に大して感謝の気持ちのないくせに、取捨選択する度胸がなく、差し障りのないように機械的に羅列しただけだろう」という印象を持ち、逆に謝辞としての意味が薄れます。

絶対に謝辞に含めなくてはいけないのは、およそ次の通りです\footnote{これは投稿論文にも当てはまりますが、投稿論文では、貢献の大きい人は共著者になるべきです。}。順序は前後しても構いませんが、貢献度の高い人を前にするべきです。
\begin{enumerate}
\item 指導教員(教授、准教授など)
\item 修士論文の研究テーマやアイデアを考えた人、提供した人(指導教員の場合が多い)
\item 指導教員以外で直接的に指導した人(助教、ポスドクなど)
\item 論文において本質的となる議論や指摘を行ってくれた人
\item 実験やデータ解析をかなり手伝ってくれた人(先輩や共同研究者など)
\item データや解析スクリプトなどを提供してくれた人(共同研究者など)
\item 研究資金を受給していれば、その資金名と提供元(学内の研究支援も含む)
\end{enumerate}

修士論文の場合、さらに同期の学生や事務補佐員などを謝辞に加える学生も多くいます。これは科学論文として必須ではありませんので、謝辞に加えないからといって失礼に当たるわけではありません。また家族への謝辞を加えることもよくあります\footnote{独立生計でない場合、研究資金の提供先ですので当然と言えば当然かもしれません。}。

また謝辞では次のことに注意してください。

\begin{enumerate}
\item 現在は指導教「員」と呼称し、指導教「官」ではない\footnote{国立大学法人化によって、大学教員は公務員ではなくなったため。}。
\item 職階を間違えないこと。「准教授」を「助教授」、「助教」を「助手」とする間違いが多い\footnote{2007年に名称が変更となった。}。
\item 氏名の漢字を絶対に間違えない。旧字体(曉と暁など)や異体字(齋藤と斎藤など)に気をつけること。
\item 所属先は略称ではなく正式名称を書くこと。大学の研究所などの場合、研究所名だけでなく大学名を先に書くこと。
\item 敬称をつけること(「2年間にわたり御指導くださった XX 教授」など)。博士号を持っている人には「博士」をつけること。
\end{enumerate}


\renewcommand{\bibname}{引用文献}
\bibliographystyle{style/jecon}
\bibliography{thesis}
\label{page:bib}

\end{document}
