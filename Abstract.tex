ここには論文の概要(abstract)を書きます。論文の先頭なので早い時期に書き始める人がいますが、論文の結論や論理展開はなかなか執筆終盤まで固まりません。そのため、論文の流れや結論がかなり明確になった最終段階で書くようにしましょう。

概要は論文全体の内容を短文で説明するものですので、研究の背景と目的、研究内容、結果と結論などが全て網羅されている必要があります。ここを読んだだけで、論文の中身が大雑把に把握できるようにすることが大切です。原則として改行せずに1段落で書きますが、これは複数段落に分けて書くような文章を無理やり1段落に合体させろということではありません。1段落で流れるように書いてください。文量としてはA4の半分から3分の2程度だと思います。2ページにもわたる概要はありえません。

この概要と序論を同一視する人が多く見られます。おそらく修論提出の直前になって慌てて書くため、序論を要約する格好になってしまうのでしょう。しかし序論には研究結果の詳細や、それに対しての議論や考察は書かれていないことが一般的です。そのため序論の焼き直しのような概要を書かれてしまうと、論文の結論や議論を概要で追うことができなくなり概要を書く意味が薄れてしまいますので、「論文全体の内容を短くまとめているか」に注意して要約してください。
