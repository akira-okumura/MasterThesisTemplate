\chapter*{付録} % * を付すことで、章番号を出さなくする
\addcontentsline{toc}{chapter}{付録} % 目次に載せる

「付録」(appendix)は、研究内容の細かい部分を詳述する場合、論文の本文に載せるには少し脱線しており論理の流れを阻害する情報を別の場所にまとめたい場合、論文に必須ではないものの読者にとって有益となる、もしくは理解を助けるような情報を載せる場合などに使用します。付録を必要としない論文ももちろん存在しますので、そこは著者の判断です。

例えば、たくさんの観測データを様々なモデルでフィットした場合には似たようなフィット結果の絵がたくさん出てくるはずです。そのような図は本文中に大量に出されても大切な情報を見失ってしまいますので、その一部のみを本文中に載せ、他の大部分を付録に載せることが推奨されます。他には、何かしらの長い式変形や証明を載せる必要がある場合、付録に移動する場合があります。

修士論文は学部実験で書いてきた実験レポートに比べるとかなり文章の長いものです。そのため論理展開に重要ではない図や計算が途中でたくさん出てきてしまうと、論文を執筆している本人は論理展開をしにくくなり、また読者は話の流れを見落としやすくなり両者ともに脱線しがちです。付録に図や計算を積極的に移動することで、話の流れを整理すると効果的です。

% 付録は chapter の 1 つとして作りますが、章番号は表示しません。
% また付録の 1 つずつはアルファベットで番号付けをするのが一般的です。
\setcounter{section}{0} % section の番号をゼロにリセットする
\renewcommand{\thesection}{\Alph{section}} % 数字ではなくアルファベットで数える
\setcounter{equation}{0} % 式番号を A.1 のようにする
\renewcommand{\theequation}{\Alph{section}.\arabic{equation}}
\setcounter{figure}{0} % 図番号
\renewcommand{\thefigure}{\Alph{section}.\arabic{figure}}
\setcounter{table}{0} % 表番号
\renewcommand{\thetable}{\Alph{section}.\arabic{table}}

\section{長い証明}
\label{sec:very_long_proof}
式~\eqref{eq}のように、式番号がアルファベットとアラビア数字の組み合わせになるように\LaTeX{}ソース中で設定してありますので、中身を眺めてみてください。またこの節も他と同様「付録~\ref{sec:very_long_proof}」のように参照することができます。

\begin{equation}
  \label{eq}
  1 + 1 = 2
\end{equation}

\section{たくさんのフィットの図}
